\documentclass[a4paper, 14pt]{extarticle}

 
% Поля
%--------------------------------------
\usepackage{minted}
\usepackage{xcolor}
\usepackage{geometry}
\usepackage{float}
\geometry{a4paper,tmargin=2cm,bmargin=2cm,lmargin=3cm,rmargin=1cm}
%--------------------------------------


%Russian-specific packages
%--------------------------------------
\usepackage[T2A]{fontenc}
\usepackage[utf8]{inputenc} 
\usepackage[english, main=russian]{babel}
%--------------------------------------

\usepackage{textcomp}

% Красная строка
%--------------------------------------
\usepackage{indentfirst}               
%--------------------------------------             


%Graphics
%--------------------------------------
\usepackage{graphicx}
\graphicspath{ {./images/} }
\usepackage{wrapfig}
%--------------------------------------

% Полуторный интервал
%--------------------------------------
\linespread{1.3}                    
%--------------------------------------

%Выравнивание и переносы
%--------------------------------------
% Избавляемся от переполнений
\sloppy
% Запрещаем разрыв страницы после первой строки абзаца
\clubpenalty=10000
% Запрещаем разрыв страницы после последней строки абзаца
\widowpenalty=10000
%--------------------------------------

%Списки
\usepackage{enumitem}

%Подписи
\usepackage{caption} 

%Гиперссылки
\usepackage{hyperref}

\hypersetup {
	unicode=true
}

%Рисунки
%--------------------------------------
\DeclareCaptionLabelSeparator*{emdash}{~--- }
\captionsetup[figure]{labelsep=emdash,font=onehalfspacing,position=bottom}
%--------------------------------------

\usepackage{tempora}
\usepackage{amsmath}
\usepackage{color}
\usepackage{listings}
\lstset{
  belowcaptionskip=1\baselineskip,
  breaklines=true,
  frame=L,
  xleftmargin=\parindent,
  language=C++,
  showstringspaces=false,
  basicstyle=\footnotesize\ttfamily,
  keywordstyle=\bfseries\color{blue},
  commentstyle=\itshape\color{purple},
  identifierstyle=\color{black},
  stringstyle=\color{red},
  extendedchars=\true,
}

%--------------------------------------
%			НАЧАЛО ДОКУМЕНТА
%--------------------------------------

\begin{document}

%--------------------------------------
%			ТИТУЛЬНЫЙ ЛИСТ
%--------------------------------------
\begin{titlepage}
\thispagestyle{empty}
\newpage


%Шапка титульного листа
%--------------------------------------
\vspace*{-60pt}
\hspace{-65pt}
\begin{minipage}{0.3\textwidth}
\hspace*{-20pt}\centering
\includegraphics[width=\textwidth]{emblem.png}
\end{minipage}
\begin{minipage}{0.67\textwidth}\small \textbf{
\vspace*{-0.7ex}
\hspace*{-6pt}\centerline{Министерство науки и высшего образования Российской Федерации}
\vspace*{-0.7ex}
\centerline{Федеральное государственное бюджетное образовательное учреждение }
\vspace*{-0.7ex}
\centerline{высшего образования}
\vspace*{-0.7ex}
\centerline{<<Московский государственный технический университет}
\vspace*{-0.7ex}
\centerline{имени Н.Э. Баумана}
\vspace*{-0.7ex}
\centerline{(национальный исследовательский университет)>>}
\vspace*{-0.7ex}
\centerline{(МГТУ им. Н.Э. Баумана)}}
\end{minipage}
%--------------------------------------

%Полосы
%--------------------------------------
\vspace{-25pt}
\hspace{-35pt}\rule{\textwidth}{2.3pt}

\vspace*{-20.3pt}
\hspace{-35pt}\rule{\textwidth}{0.4pt}
%--------------------------------------

\vspace{1.5ex}
\hspace{-35pt} \noindent \small ФАКУЛЬТЕТ\hspace{80pt} <<Информатика и системы управления>>

\vspace*{-16pt}
\hspace{47pt}\rule{0.83\textwidth}{0.4pt}

\vspace{0.5ex}
\hspace{-35pt} \noindent \small КАФЕДРА\hspace{50pt} <<Теоретическая информатика и компьютерные технологии>>

\vspace*{-16pt}
\hspace{30pt}\rule{0.866\textwidth}{0.4pt}
  
\vspace{11em}

\begin{center}
\Large {\bf Лабораторная работа № 1} \\ 
\large {\bf по курсу <<Операционные системы>>} \\ 
{ReactOS и NetBSD. Среда сборки, установки и
тестирование в виртуальной машине} \\
\end{center}\normalsize

\vspace{8em}


\begin{flushright}
  {Студент группы ИУ9-42Б Нащекин Н. Д.\hspace*{15pt} \\
  \vspace{2ex}
  Преподаватель: Брагин А. В.\hspace*{15pt}}
\end{flushright}

\bigskip

\vfill
 

\begin{center}
\textsl{Москва, 2024}
\end{center}
\end{titlepage}
%--------------------------------------
%		КОНЕЦ ТИТУЛЬНОГО ЛИСТА
%--------------------------------------

\renewcommand{\ttdefault}{pcr}

\setlength{\tabcolsep}{3pt}
\newpage
\setcounter{page}{2}

\section{Содержание}
\begin{flushleft}
3 - Цель \newline
4 - Постановка задачи \newline
5 - Практическая реализация \newline
8 - Результаты \newline
10 - Выводы \newline
11 - Список литературы \newline
\end{flushleft}
\pagebreak

\section{Цель}
\begin{flushleft}
Приобретение навыков по самостоятельной сборке «с нуля»
дистрибутивов операционных систем ReactOS и NetBSD с последующей их
установкой в виртуальную машину.

\end{flushleft}
\pagebreak

\section{Постановка задачи}
\begin{flushleft}
ЧАСТЬ 1. СБОРКА ДИСТРИБУТИВА REACTOS \newline
Настроить среду сборки и тестирования ReactOS на своём компьютере
(или на компьютере в учебной аудитории университета), собрать установочный
образ и произвести его установку в виртуальную машину с помощью
программных средств виртуализации. \newline

ЧАСТЬ 2. ОПЕРАЦИОННАЯ СИСТЕМА NETBSD: СРЕДА СБОРКИ,
УСТАНОВКА И ТЕСТИРОВАНИЕ \newline
Установить новый выпуск NetBSD из дистрибутива с официального сайта
проекта NetBSD https://www.netbsd.org/
в виртуальную машину. Настроить среду сборки и тестирования NetBSD в виртуальной машине на своём компьютере (или на компьютере в учебной аудитории университета). Пересобрать ядро в этой виртуальной машине, добавив вывод фамилии
обучающегося в debug log.

\end{flushleft}
\pagebreak

\section{Практическая реализация}
\begin{flushleft}
ЧАСТЬ 1 \newline
Сначала мной была установлена ReactOS Build Environment (RosBE)[1]. После установки среды сборки я получил исходный код операционной системы[2]. Используя среду сборки, я собрал ISO-образ ReactOS и установил его в вирутальной машине. В настройках виртуальной машины я включил последовательный порт COM-1 и перенаправил вывод в файл для удобного изучения логов. \newline
Далее сразу в нескольких файлах ядра (ntoskrnl) я добавил вывод своих фамилии и имени в лог, например в файле ntoskrnl/ex/init.c:
\end{flushleft}
\begin{minted}[frame=lines,framesep=2mm,baselinestretch=1.2,fontsize=\footnotesize,linenos]{cpp}
/* Initialize keyed events */
    if (ExpInitializeKeyedEventImplementation() == FALSE)
    {
        DPRINT1("Executive: Keyed event initialization failed\n");
        return FALSE;
    }

    /* Initialize Win32K */
    if (ExpWin32kInit() == FALSE)
    {
        DPRINT1("Executive: Win32 initialization failed\n");
        return FALSE;
    }
    
    DPRINT1("NASHCHEKIN NIKITA\n");
    
    
    return TRUE;
\end{minted}

\begin{flushleft}
В файле ntoskrnl/io/iomgr/driver.c:
\end{flushleft}

\begin{minted}[frame=lines,framesep=2mm,baselinestretch=1.2,fontsize=\footnotesize,linenos]{cpp}
IopDeleteDriver(IN PVOID ObjectBody)
{
    PDRIVER_OBJECT DriverObject = ObjectBody;
    PIO_CLIENT_EXTENSION DriverExtension, NextDriverExtension;
    PAGED_CODE();

    DPRINT1("Deleting driver object '%wZ'\n", &DriverObject->DriverName);
    DPRINT1("NASHCHEKIN NIKITA!!!\n");

    /* There must be no device objects remaining at this point */
    ASSERT(!DriverObject->DeviceObject);

    /* Get the extension and loop them */
    DriverExtension = IoGetDrvObjExtension(DriverObject)->ClientDriverExtension;
    while (DriverExtension)
    {
        /* Get the next one */
        NextDriverExtension = DriverExtension->NextExtension;
        ExFreePoolWithTag(DriverExtension, TAG_DRIVER_EXTENSION);

        /* Move on */
        DriverExtension = NextDriverExtension;
    }

    /* Check if the driver image is still loaded */
    if (DriverObject->DriverSection)
    {
        /* Unload it */
        MmUnloadSystemImage(DriverObject->DriverSection);
    }

    /* Check if it has a name */
    if (DriverObject->DriverName.Buffer)
    {
        /* Free it */
        ExFreePool(DriverObject->DriverName.Buffer);
    }

    /* Check if it has a service key name */
    if (DriverObject->DriverExtension->ServiceKeyName.Buffer)
    {
        /* Free it */
        ExFreePool(DriverObject->DriverExtension->ServiceKeyName.Buffer);
    }
}
\end{minted}

\begin{flushleft}
Аналогичные строки я добавил в нескольких других файлах (для надежности!). Далее я пересобрал ядро и переустановил систему. \newline

ЧАСТЬ 2 \newline
Сначала я скачал образ дистрибутива ОС NetBSD[3] и установил его в виртуальную машину. При установке системы я не настроил использование com-порта, поэтому пришлось добавить в файл /boot.cfg строку \newline 
consdev=com0,115200 \newline
Аналогичным образом настроил вывод отладки в файл. Затем по ftp (из NetBSD) скачал архив с ядром системы[4], распаковал его, изменил несколько файлов, добавив вывод своей фамилии, и собрал ядро[5]. Заменив ядро, убедился в наличии моих фамилии и имени в логах.
Фрагмент кода из файла usr/src/sys/kern/init\char`_ main.c:
\end{flushleft}

\begin{minted}[frame=lines,framesep=2mm,baselinestretch=1.2,fontsize=\footnotesize,linenos]{cpp}
/* Create the aiodone daemon kernel thread. */
	if (workqueue_create(&uvm.aiodone_queue, "aiodoned",
	    uvm_aiodone_worker, NULL, PRI_VM, IPL_NONE, WQ_MPSAFE))
		panic("fork aiodoned");

	printf("NASHCHEKIN NIKITA!!!");
	/* Mount the root file system. */
	do {
		domountroothook(root_device);
		if ((error = vfs_mountroot())) {
			printf("cannot mount root, error = %d\n", error);
			boothowto |= RB_ASKNAME;
			setroot(root_device,
			    (rootdev != NODEV) ? DISKPART(rootdev) : 0);
		}
	} while (error != 0);
	mountroothook_destroy();
\end{minted}
\pagebreak

\section{Результаты}
\begin{flushleft}
Был реализован вывод моих фамилии и имени в лог при запуске операционных систем ReactOS и NetBSD. 

\begin{figure}[H]

\centering

\includegraphics[width=0.8\linewidth]{1.png}

\caption{Скриншот 1}

\label{fig:mpr}

\end{figure}

\begin{figure}[H]

\centering

\includegraphics[width=0.8\linewidth]{2.png}

\caption{Скриншот 2}

\label{fig:mpr}

\end{figure}

\begin{figure}[H]

\centering

\includegraphics[width=0.8\linewidth]{3.png}

\caption{Скриншот 3}

\label{fig:mpr}

\end{figure}

\begin{figure}[H]

\centering

\includegraphics[width=0.8\linewidth]{4.png}

\caption{Скриншот 4}

\label{fig:mpr}

\end{figure}

\begin{figure}[H]

\centering

\includegraphics[width=0.8\linewidth]{5.png}

\caption{Скриншот 5}

\label{fig:mpr}

\end{figure}

\end{flushleft}
\pagebreak 

\section{Выводы}
Выполнив данную лабораторную работу, я приобрёл навыки сборки с нуля дистрибутивов операционных систем и проведения их установки. Я смог найти файлы, отвечающие за инициализацию операционных систем, и добавил туда свои строки. Выполнять первую часть работы на ReactOS было проще: я мог изменить файлы из старшей системы, провести компиляцию ядра удобной средой сборки и просто переустановить систему. Во второй части мне пришлось вручную компилировать ядро внутри NetBSD, что немного сложнее.
\pagebreak

\section{Список литературы}
\begin{flushleft}
[1] - https://reactos.org/wiki/Building\char`_ ReactOS\newline
[2] - https://github.com/reactos/reactos.git\newline
[3] - https://www.netbsd.org/\newline
[4] - http://ftp.NetBSD.org/pub/NetBSD/NetBSD-9.3/source/sets/syssrc.tgz\newline
[5] - https://netbsd.org/docs/guide/en/chap-kernel.html\char`# chap-kernel-building-manually\newline
\end{flushleft}
\end{document}
