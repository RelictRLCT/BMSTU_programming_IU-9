\documentclass[a4paper, 14pt]{extarticle}

 
% Поля
%--------------------------------------
\usepackage{minted}
\usepackage{xcolor}
\usepackage{geometry}
\usepackage{float}
\geometry{a4paper,tmargin=2cm,bmargin=2cm,lmargin=3cm,rmargin=1cm}
%--------------------------------------


%Russian-specific packages
%--------------------------------------
\usepackage[T2A]{fontenc}
\usepackage[utf8]{inputenc} 
\usepackage[english, main=russian]{babel}
%--------------------------------------

\usepackage{textcomp}

% Красная строка
%--------------------------------------
\usepackage{indentfirst}               
%--------------------------------------             


%Graphics
%--------------------------------------
\usepackage{graphicx}
\graphicspath{ {./images/} }
\usepackage{wrapfig}
%--------------------------------------

% Полуторный интервал
%--------------------------------------
\linespread{1.3}                    
%--------------------------------------

%Выравнивание и переносы
%--------------------------------------
% Избавляемся от переполнений
\sloppy
% Запрещаем разрыв страницы после первой строки абзаца
\clubpenalty=10000
% Запрещаем разрыв страницы после последней строки абзаца
\widowpenalty=10000
%--------------------------------------

%Списки
\usepackage{enumitem}

%Подписи
\usepackage{caption} 

%Гиперссылки
\usepackage{hyperref}

\hypersetup {
	unicode=true
}

%Рисунки
%--------------------------------------
\DeclareCaptionLabelSeparator*{emdash}{~--- }
\captionsetup[figure]{labelsep=emdash,font=onehalfspacing,position=bottom}
%--------------------------------------

\usepackage{tempora}
\usepackage{amsmath}
\usepackage{color}
\usepackage{listings}
\lstset{
  belowcaptionskip=1\baselineskip,
  breaklines=true,
  frame=L,
  xleftmargin=\parindent,
  language=C++,
  showstringspaces=false,
  basicstyle=\footnotesize\ttfamily,
  keywordstyle=\bfseries\color{blue},
  commentstyle=\itshape\color{purple},
  identifierstyle=\color{black},
  stringstyle=\color{red},
  extendedchars=\true,
}

%--------------------------------------
%			НАЧАЛО ДОКУМЕНТА
%--------------------------------------

\begin{document}

%--------------------------------------
%			ТИТУЛЬНЫЙ ЛИСТ
%--------------------------------------
\begin{titlepage}
\thispagestyle{empty}
\newpage


%Шапка титульного листа
%--------------------------------------
\vspace*{-60pt}
\hspace{-65pt}
\begin{minipage}{0.3\textwidth}
\hspace*{-20pt}\centering
\includegraphics[width=\textwidth]{emblem.png}
\end{minipage}
\begin{minipage}{0.67\textwidth}\small \textbf{
\vspace*{-0.7ex}
\hspace*{-6pt}\centerline{Министерство науки и высшего образования Российской Федерации}
\vspace*{-0.7ex}
\centerline{Федеральное государственное бюджетное образовательное учреждение }
\vspace*{-0.7ex}
\centerline{высшего образования}
\vspace*{-0.7ex}
\centerline{<<Московский государственный технический университет}
\vspace*{-0.7ex}
\centerline{имени Н.Э. Баумана}
\vspace*{-0.7ex}
\centerline{(национальный исследовательский университет)>>}
\vspace*{-0.7ex}
\centerline{(МГТУ им. Н.Э. Баумана)}}
\end{minipage}
%--------------------------------------

%Полосы
%--------------------------------------
\vspace{-25pt}
\hspace{-35pt}\rule{\textwidth}{2.3pt}

\vspace*{-20.3pt}
\hspace{-35pt}\rule{\textwidth}{0.4pt}
%--------------------------------------

\vspace{1.5ex}
\hspace{-35pt} \noindent \small ФАКУЛЬТЕТ\hspace{80pt} <<Информатика и системы управления>>

\vspace*{-16pt}
\hspace{47pt}\rule{0.83\textwidth}{0.4pt}

\vspace{0.5ex}
\hspace{-35pt} \noindent \small КАФЕДРА\hspace{50pt} <<Теоретическая информатика и компьютерные технологии>>

\vspace*{-16pt}
\hspace{30pt}\rule{0.866\textwidth}{0.4pt}
  
\vspace{11em}

\begin{center}
\Large {\bf Лабораторная работа № 4} \\ 
\large {\bf по курсу <<Операционные системы>>} \\ 
{РАБОТА С ВИРТУАЛЬНОЙ ПАМЯТЬЮ} \\
\end{center}\normalsize

\vspace{8em}


\begin{flushright}
  {Студент группы ИУ9-42Б Нащекин Н. Д.\hspace*{15pt} \\
  \vspace{2ex}
  Преподаватель: Брагин А. В.\hspace*{15pt}}
\end{flushright}

\bigskip

\vfill
 

\begin{center}
\textsl{Москва, 2024}
\end{center}
\end{titlepage}
%--------------------------------------
%		КОНЕЦ ТИТУЛЬНОГО ЛИСТА
%--------------------------------------

\renewcommand{\ttdefault}{pcr}

\setlength{\tabcolsep}{3pt}
\newpage
\setcounter{page}{2}

\section{Содержание}
\begin{flushleft}
3 - \hyperref[sec:Goal]{Цель}\newline
4 - \hyperref[sec:ToDo]{Постановка задачи}\newline
5 - \hyperref[sec:Realization]{Практическая реализация}\newline
11 - \hyperref[sec:Results]{Результаты}\newline
13 - \hyperref[sec:Conclusion]{Выводы}\newline
14 - \hyperref[sec:Literature]{Список литературы}\newline
\end{flushleft}
\pagebreak

\section{Цель} \label{sec:Goal}
\begin{flushleft}
Разработать загружаемый по требованию модуль ядра
(драйвер) для операционных систем ReactOS / Windows и NetBSD, работающий
с виртуальной памятью.

\end{flushleft}
\pagebreak

\section{Постановка задачи} \label{sec:ToDo}
\begin{flushleft}
ЧАСТЬ 1. РАБОТА С ПАМЯТЬЮ В РЕЖИМЕ ЯДРА В ОС СЕМЕЙСТВА
WINDOWS NT \newline \newline
Используя созданный в лабораторной работе № 3 минимальный драйвер,
совместимый с операционными системами Windows NT / ReactOS, реализовать
следующие действия с виртуальной памятью:\newline
1. Зарезервировать 10 страниц виртуальной памяти, используя
ZwAllocateVirtualMemory(NtCurrentProcess(), MEM\char`_ RESERVE, …) \newline
2. Обеспечить 5 первых страниц из выделенных 10-ти физическими
страницами
памяти,
используя
ZwAllocateVirtualMemory(NtCurrentProcess(),
MEM\char`_ COMMIT, …).\newline
3. Вывести физические адреса и значения PTE для этих 5 страниц в
шестнадцатеричном формате.\newline
4. Освободить выделенную память.\newline

ЧАСТЬ 2. РАБОТА С ПАМЯТЬЮ В РЕЖИМЕ ЯДРА В NETBSD \newline \newline
Используя созданный в лабораторной работе № 3 драйвер, реализовать
следующие действия с виртуальной памятью: \newline
1.
Зарезервировать
10
страниц
виртуальной
памяти
используя
функцию ядра uvm\char`_ km\char`_ alloc().\newline
2.
страницами
Обеспечить 5 первых страниц из выделенных 10-ти физическими
памяти
используя
функции
ядра
uvm\char`_ pglistalloc()
и
pmap\char`_ kenter\char`_ pa().\newline
3.
Вывести физические адреса и значения PTE для этих 5 страниц в
шестнадцатеричном формате.\newline
4.
Освободить
выделенную
память
используя
функции
ядра
pmap\char`_ kremove(), pmap\char`_ update(), uvm\char`_ pglistfree(), uvm\char`_ km\char`_ free().\newline

\end{flushleft}
\pagebreak

\section{Практическая реализация} \label{sec:Realization}
\begin{flushleft}
ЧАСТЬ 1 \newline
Аналогично лабораторной работе 3, в скачанном ранее каталоге с исходным кодом ReactOS[1] в папке reactos/drivers я создал папку lab4\char`_ driver. В drivers/CMakeLists.txt добавил строку 
\end{flushleft}

\begin{minted}[frame=lines,framesep=2mm,baselinestretch=1.2,fontsize=\footnotesize,linenos]{cpp}
add_subdirectory(lab4_driver)
\end{minted}

\begin{flushleft}
 для того, чтобы мой драйвер участвовал в сборке
системы. В папке drivers/lab4\char`_ driver были созданы три файла: CMakeLists.txt, lab4\char`_ driver.c и lab4\char`_ driver.rc. Содержимое этих файлов было создано на основе моего драйвера из прошлых лабораторных. Содержимое CMakeLists.txt:
\end{flushleft}

\begin{minted}[frame=lines,framesep=2mm,baselinestretch=1.2,fontsize=\footnotesize,linenos]{cpp}
add_library(lab4_driver MODULE lab4_driver.c lab4_driver.rc)
set_module_type(lab4_driver kernelmodedriver)
add_importlibs(lab4_driver ntoskrnl)
add_cd_file(TARGET lab4_driver DESTINATION reactos/system32/drivers FOR all)
\end{minted}

\begin{flushleft}
В главном файле с помощью ZwAllocateVirtualMemory[2] выделяется 10 страниц виртуальной памяти. Затем для 5 из 10 страниц с помощью этой же функции выделяются 5 страниц физической памяти. С помощью битовых операций в цикле для каждой страницы вычисляется значение PTE и выводятся некоторые его поля: Valid, Dirty, LargePage, Accessed, reserved и физический адрес. Также для проверки выводится физический адрес, вычисленный не с помощью PTE, а полученный с использованием MmGetPhysicalAddress. После вывода зарезервированная память освобождается.\newline
lab4\char`_ driver.c:[3]
\end{flushleft}

\begin{minted}[frame=lines,framesep=2mm,baselinestretch=1.2,fontsize=\footnotesize,linenos]{cpp}
#include <ntddk.h>
#include <debug.h>
#include <ntifs.h>
#include <ndk/exfuncs.h>
#include <ndk/ketypes.h>
#include <ntstrsafe.h>

static DRIVER_UNLOAD DriverUnload;
static VOID NTAPI
DriverUnload(IN PDRIVER_OBJECT DriverObject)
{
    IoDeleteDevice(DriverObject->DeviceObject);
	DPRINT1("NASHCHEKIN NIKITA'S Driver unloaded\n");
}

NTSTATUS NTAPI DriverEntry(IN PDRIVER_OBJECT DriverObject,
            IN PUNICODE_STRING RegistryPath) {
    /* For non-used parameter */
    UNREFERENCED_PARAMETER(RegistryPath);

    /* Setup the Driver Object */
    DriverObject->DriverUnload = DriverUnload;

	///Основная функция
    DPRINT1("NASHCHEKIN NIKITA!!! From another driver\n\n");

    PVOID base_adress = 0;
    long unsigned int region_size = PAGE_SIZE * 10;

    int status = ZwAllocateVirtualMemory(NtCurrentProcess(), &base_adress, 0, &region_size, MEM_RESERVE,  PAGE_READWRITE);

    if (!NT_SUCCESS(status)) {
        DPRINT1("Ошибка при выделении памяти");
        goto end;
    }

    long unsigned int reserved_region_size = PAGE_SIZE * 5;

    status = ZwAllocateVirtualMemory(NtCurrentProcess(), &base_adress, 0, &reserved_region_size, MEM_COMMIT,  PAGE_READWRITE);

    if (!NT_SUCCESS(status)) {
        DPRINT1("Ошибка при резервировании памяти");
        goto end;
    }

    for (int i = 0; i < 5; i++) {
        DPRINT1("Страница %d:\n", i+1);

        PVOID page_virtual_address = (PVOID)((ULONG_PTR)base_adress + i * PAGE_SIZE);


        PMDL mdl = IoAllocateMdl(page_virtual_address, PAGE_SIZE, FALSE, FALSE, NULL);
        if (mdl == NULL) {
            DPRINT1("Ошибка при создании MDL\n");
            goto end;
        }

        MmProbeAndLockPages(mdl, KernelMode, IoReadAccess);
        PHYSICAL_ADDRESS physical_address = MmGetPhysicalAddress(page_virtual_address);
        MmUnlockPages(mdl);
        IoFreeMdl(mdl);

       PHARDWARE_PTE_X86 pte = (PHARDWARE_PTE_X86)((ULONG)0xC0000000 + (((ULONG)page_virtual_address & 0x003FFFFF) >> 10));


        if (pte != NULL) {
            DPRINT1("Виртуальный адрес: %#x\n", page_virtual_address);
            DPRINT1("Физический адрес: %#x\n", physical_address);

            DPRINT1("PTE (Valid): %s\n", (pte->Valid != 0 ? "Да" : "Нет"));
            DPRINT1("PTE (Dirty): %s\n", (pte->Dirty != 0 ? "Да" : "Нет"));
            DPRINT1("PTE (LargePage): %s\n", (pte->LargePage != 0 ? "Да" : "Нет"));
            DPRINT1("PTE (Accessed): %s\n", (pte->Accessed != 0 ? "Да" : "Нет"));
            DPRINT1("PTE (Reserved): %s\n", (pte->reserved != 0 ? "Да" : "Нет"));
            DPRINT1("Физический адрес из PTE (для проверки): %#x\n\n", (pte->PageFrameNumber)<<12);
        } else {
            DPRINT1("PTE не получен\n\n");
        }
    }

    for (int i = 5; i < 10; i++) {
        DPRINT1("Страница %d:\n", i+1);

        PVOID page_virtual_address = (PVOID)((ULONG_PTR)base_adress + i * PAGE_SIZE);

        PHARDWARE_PTE_X86 pte = (PHARDWARE_PTE_X86)((ULONG)0xC0000000 + (((ULONG)page_virtual_address & 0x003FFFFF) >> 10));

        DPRINT1("Виртуальный адрес: %#x\n", page_virtual_address);
        DPRINT1("Физический адрес: %#x\n", 000000);
        DPRINT1("PTE (Valid): %s\n", (pte->Valid != 0 ? "Да" : "Нет"));
        DPRINT1("PTE (Dirty): %s\n", (pte->Dirty != 0 ? "Да" : "Нет"));
        DPRINT1("PTE (LargePage): %s\n", (pte->LargePage != 0 ? "Да" : "Нет"));
        DPRINT1("PTE (Accessed): %s\n", (pte->Accessed != 0 ? "Да" : "Нет"));
        DPRINT1("PTE (Reserved): %s\n", (pte->reserved != 0 ? "Да" : "Нет"));
        DPRINT1("Физический адрес из PTE (для проверки): %#x\n\n", (pte->PageFrameNumber)<<12);
    }

    status = ZwFreeVirtualMemory(NtCurrentProcess(), &base_adress, &reserved_region_size, MEM_DECOMMIT);

    if (!NT_SUCCESS(status)) {
        DPRINT1("Ошибка при освобождении физической памяти\n");
        goto end;
    }

    status = ZwFreeVirtualMemory(NtCurrentProcess(), &base_adress, &region_size, MEM_RELEASE);

    if (!NT_SUCCESS(status)) {
        DPRINT1("Ошибка при освобождении виртуальной памяти\n");
        goto end;
    }

    DPRINT1("Память освобождена!\n\n");

    DPRINT1("%d\n", base_adress);
    DPRINT1("%d\n", region_size);
    DPRINT1("%d\n", reserved_region_size);
    DPRINT1("!!!");

    end:
    /* Return success. */
    return STATUS_SUCCESS;
}

\end{minted}

\begin{flushleft}
lab4\char`_ driver.rc:
\end{flushleft}

\begin{minted}[frame=lines,framesep=2mm,baselinestretch=1.2,fontsize=\footnotesize,linenos]{cpp}
#define REACTOS_VERSION_DLL
#define REACTOS_STR_FILE_DESCRIPTION  "my another driver"
#define REACTOS_STR_INTERNAL_NAME     "lab4_driver"
#define REACTOS_STR_ORIGINAL_FILENAME "lab4_driver.sys"
#include <reactos/version.rc>
\end{minted}

\begin{flushleft}
Далее в среде сборки был пересобран образ, а затем переустановлена система. В работающей системе были выполнены команды:
\end{flushleft}

\begin{minted}[frame=lines,framesep=2mm,baselinestretch=1.2,fontsize=\footnotesize,linenos ,escapeinside=||]{cpp}

sc create lab4driver binPath=C:|\textbackslash|ReactOS|\textbackslash|system32|\textbackslash|drivers|\textbackslash|lab4_driver.sys type= kernel
sc start lab4driver

\end{minted}

\begin{flushleft}
В логи выводится информация о всех 10 страницах виртуальной памяти. \newline \newline
ЧАСТЬ 2 \newline
Как и в предыдущей лабораторной, сначала был создан файл /usr/src/sys/dev/lab4.c и добавлена реализация драйвера:
\end{flushleft}

\begin{minted}[frame=lines,framesep=2mm,baselinestretch=1.2,fontsize=\footnotesize,linenos]{cpp}
#include <sys/module.h>
#include <sys/kernel.h>
#include <sys/proc.h>
#include <sys/types.h>
#include <sys/systm.h>
#include <uvm/uvm.h>
#include <sys/param.h>

MODULE(MODULE_CLASS_MISC, lab4, NULL);

static int lab4_modcmd(modcmd_t cmd, void* arg){
	printf("NASHCHEKIN NIKITA Frrom another NetBSD driver!!!\n\n");

	vaddr_t base_addres = uvm_km_alloc(kernel_map, 10 * 4096, 0, 
				UVM_KMF_VAONLY | UVM_KMF_WAITVA);
	paddr_t physical_page_addres;
	struct pglist plist;

	if (uvm_pglistalloc(5 * 4096, 0, -1, 0, -1, &plist, 5, 0) != 0) {
		printf("Allocation of phys page failed\n");
		return 1;
	}	

	struct vm_page* pg = TAILQ_FIRST(&plist);
	
	for (int i = 0; i < 5; i++) {

		physical_page_addres = VM_PAGE_TO_PHYS(pg);
		
		pmap_kenter_pa(base_addres + i*4096, physical_page_addres, 
			VM_PROT_READ | VM_PROT_WRITE, 0);
		
		pg = TAILQ_NEXT(pg, pageq.queue);
		printf("PHYS ADDR: %#lx\n", physical_page_addres);
	
	}


	for (int i = 0; i < 10; i++) {
	
		printf("\nPage %d:\n", i+1);
		printf("Virtual addres: %#lx\n", base_addres + i*4096);
		
		pt_entry_t* pte;
		pte = vtopte(base_addres + i*4096);
		
		printf("Valid: %s\n", ((*pte)&PG_V) ? "true" : "false");
		printf("Used: %s\n", ((*pte)&PG_U) ? "true" : "false");
		printf("Modified: %s\n", ((*pte)&PG_M) ? "true" : "false");
		printf("Physical addres: %#lx\n", (*pte)&PG_FRAME); 
	}

	pmap_kremove(base_addres, 10 * 4096);
	pmap_update(pmap_kernel());	
	uvm_pglistfree(&plist);
	uvm_km_free(kernel_map, base_addres, 10 * 4096,
				 UVM_KMF_VAONLY | UVM_KMF_WAITVA);
	return 0;
}
\end{minted}

\begin{flushleft}
Драйвер резервирует 10 страниц вирутальной памяти с помощью uvm\char`_ km\char`_ alloc()[4], выделяет 5 страниц в физической памяти с помощью uvm\char`_ pglistalloc()[5] и привязывает к 5 первым страницам виртуальной памяти по одной соответствующей странице физической памяти с помощью функции pmap\char`_ kenter\char`_ pa()[6]. Затем вычисляется PTE с помощью vtopte()[7] и выводятся его поля. После вывода зарезервированная память освобождается[8].\newline
Далее был создан файл /usr/src/sys/modules/lab4/Makefile со следующим содержимым:
\end{flushleft}

\begin{minted}[frame=lines,framesep=2mm,baselinestretch=1.2,fontsize=\footnotesize,linenos ,escapeinside=||]{cpp}
.include "../Makefile.inc"
KMOD=lab4
.PATH: |\textdollar|{S}/dev
SRCS=lab4.c
.include <bsd.kmodule.mk>
\end{minted}

\begin{flushleft}
Затем была выполнена команда make и modload ./lab4.kmod \newline
В логи была выведена информациях обо всех страницах.
\end{flushleft}
\pagebreak

\section{Результаты} \label{sec:Results}
\begin{flushleft}
Были реализованы драйверы на операционных системах ReactOS и NetBSD, резервирующие 10 страниц в виртуальной и 5 в физической памяти. Пять первых страниц виртуальной памяти обеспечиваются пятью страницами в физической, а остальные остаются "как есть". Затем для всех страниц в виртуальной памяти вычисляется PTE (Page Table Entry) и выводятся некоторые его поля, включая физический адрес.

\begin{figure}[H]

\centering

\includegraphics[width=0.8\linewidth]{1.png}

\caption{Скриншот 1}

\label{fig:mpr}

\end{figure}

\begin{figure}[H]

\centering

\includegraphics[width=0.8\linewidth]{2.png}

\caption{Скриншот 2}

\label{fig:mpr}

\end{figure}

\begin{figure}[H]

\centering

\includegraphics[width=0.8\linewidth]{3.png}

\caption{Скриншот 3}

\label{fig:mpr}

\end{figure}

\begin{figure}[H]

\centering

\includegraphics[width=0.8\linewidth]{4.png}

\caption{Скриншот 4}

\label{fig:mpr}

\end{figure}

\end{flushleft}
\pagebreak 

\section{Выводы} \label{sec:Conclusion}
Выполнив данную лабораторную работу, я разработал два драйвера для ReactOS и NetBSD. Выполнение работы на ReactOS оказалось проще, чем на NetBSD, поскольку здесь функция ZwAllocateVirtualMemory() отвечает за выделение виртуальной памяти, выделение физической памяти и ее привязку к виртуальным страницам, в то время как на NetBSD за это отвечают сразу три разных функции. Я научился азам работы с виртуальной памятью, её выделению, привязке к физической памяти и её освобождению.
\pagebreak

\section{Список литературы} \label{sec:Literature}
\begin{flushleft}
[1] - https://reactos.org/wiki/Building\char`_ ReactOS\newline
[2] - https://learn.microsoft.com/en-us/windows-hardware/drivers/ddi/ntifs/nf-ntifs-zwallocatevirtualmemory\newline
[3] - https://cpp.hotexamples.com/examples/-/-/ZwAllocateVirtualMemory/cpp-zwallocatevirtualmemory-function-examples.html\newline
[4] - https://cpp.hotexamples.com/examples/-/-/uvm\char`_ km\char`_ alloc/cpp-uvm\char`_ km\char`_ alloc-function-examples.html\newline
[5] - https://cpp.hotexamples.com/examples/-/-/uvm\char`_ pglistalloc/cpp-uvm\char`_ pglistalloc-function-examples.html\newline
[6] - https://man.netbsd.org/pmap.9\newline
[7] - https://netbsd.org/docs/kernel/porting\char`_ netbsd\char`_ arm\char`_ soc.html\newline
[8] - https://edgebsd.org/index.php/manual/9/uvm\char`_ pglistfree
\end{flushleft}
\end{document}
