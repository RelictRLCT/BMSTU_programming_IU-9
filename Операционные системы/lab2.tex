\documentclass[a4paper, 14pt]{extarticle}

 
% Поля
%--------------------------------------
\usepackage{minted}
\usepackage{xcolor}
\usepackage{geometry}
\usepackage{float}
\geometry{a4paper,tmargin=2cm,bmargin=2cm,lmargin=3cm,rmargin=1cm}
%--------------------------------------


%Russian-specific packages
%--------------------------------------
\usepackage[T2A]{fontenc}
\usepackage[utf8]{inputenc} 
\usepackage[english, main=russian]{babel}
%--------------------------------------

\usepackage{textcomp}

% Красная строка
%--------------------------------------
\usepackage{indentfirst}               
%--------------------------------------             


%Graphics
%--------------------------------------
\usepackage{graphicx}
\graphicspath{ {./images/} }
\usepackage{wrapfig}
%--------------------------------------

% Полуторный интервал
%--------------------------------------
\linespread{1.3}                    
%--------------------------------------

%Выравнивание и переносы
%--------------------------------------
% Избавляемся от переполнений
\sloppy
% Запрещаем разрыв страницы после первой строки абзаца
\clubpenalty=10000
% Запрещаем разрыв страницы после последней строки абзаца
\widowpenalty=10000
%--------------------------------------

%Списки
\usepackage{enumitem}

%Подписи
\usepackage{caption} 

%Гиперссылки
\usepackage{hyperref}

\hypersetup {
	unicode=true
}

%Рисунки
%--------------------------------------
\DeclareCaptionLabelSeparator*{emdash}{~--- }
\captionsetup[figure]{labelsep=emdash,font=onehalfspacing,position=bottom}
%--------------------------------------

\usepackage{tempora}
\usepackage{amsmath}
\usepackage{color}
\usepackage{listings}
\lstset{
  belowcaptionskip=1\baselineskip,
  breaklines=true,
  frame=L,
  xleftmargin=\parindent,
  language=C++,
  showstringspaces=false,
  basicstyle=\footnotesize\ttfamily,
  keywordstyle=\bfseries\color{blue},
  commentstyle=\itshape\color{purple},
  identifierstyle=\color{black},
  stringstyle=\color{red},
  extendedchars=\true,
}

%--------------------------------------
%			НАЧАЛО ДОКУМЕНТА
%--------------------------------------

\begin{document}

%--------------------------------------
%			ТИТУЛЬНЫЙ ЛИСТ
%--------------------------------------
\begin{titlepage}
\thispagestyle{empty}
\newpage


%Шапка титульного листа
%--------------------------------------
\vspace*{-60pt}
\hspace{-65pt}
\begin{minipage}{0.3\textwidth}
\hspace*{-20pt}\centering
\includegraphics[width=\textwidth]{emblem.png}
\end{minipage}
\begin{minipage}{0.67\textwidth}\small \textbf{
\vspace*{-0.7ex}
\hspace*{-6pt}\centerline{Министерство науки и высшего образования Российской Федерации}
\vspace*{-0.7ex}
\centerline{Федеральное государственное бюджетное образовательное учреждение }
\vspace*{-0.7ex}
\centerline{высшего образования}
\vspace*{-0.7ex}
\centerline{<<Московский государственный технический университет}
\vspace*{-0.7ex}
\centerline{имени Н.Э. Баумана}
\vspace*{-0.7ex}
\centerline{(национальный исследовательский университет)>>}
\vspace*{-0.7ex}
\centerline{(МГТУ им. Н.Э. Баумана)}}
\end{minipage}
%--------------------------------------

%Полосы
%--------------------------------------
\vspace{-25pt}
\hspace{-35pt}\rule{\textwidth}{2.3pt}

\vspace*{-20.3pt}
\hspace{-35pt}\rule{\textwidth}{0.4pt}
%--------------------------------------

\vspace{1.5ex}
\hspace{-35pt} \noindent \small ФАКУЛЬТЕТ\hspace{80pt} <<Информатика и системы управления>>

\vspace*{-16pt}
\hspace{47pt}\rule{0.83\textwidth}{0.4pt}

\vspace{0.5ex}
\hspace{-35pt} \noindent \small КАФЕДРА\hspace{50pt} <<Теоретическая информатика и компьютерные технологии>>

\vspace*{-16pt}
\hspace{30pt}\rule{0.866\textwidth}{0.4pt}
  
\vspace{11em}

\begin{center}
\Large {\bf Лабораторная работа № 2} \\ 
\large {\bf по курсу <<Операционные системы>>} \\ 
{ЗАГРУЖАЕМЫЙ МОДУЛЬ ЯДРА (ДРАЙВЕР)} \\
\end{center}\normalsize

\vspace{8em}


\begin{flushright}
  {Студент группы ИУ9-42Б Нащекин Н. Д.\hspace*{15pt} \\
  \vspace{2ex}
  Преподаватель: Брагин А. В.\hspace*{15pt}}
\end{flushright}

\bigskip

\vfill
 

\begin{center}
\textsl{Москва, 2024}
\end{center}
\end{titlepage}
%--------------------------------------
%		КОНЕЦ ТИТУЛЬНОГО ЛИСТА
%--------------------------------------

\renewcommand{\ttdefault}{pcr}

\setlength{\tabcolsep}{3pt}
\newpage
\setcounter{page}{2}

\section{Содержание}
\begin{flushleft}
3 - Цель \newline
4 - Постановка задачи \newline
5 - Практическая реализация \newline
7 - Результаты \newline
8 - Выводы \newline
9 - Список литературы \newline
\end{flushleft}
\pagebreak

\section{Цель}
\begin{flushleft}
Разработать простой загружаемый модуль ядра (драйвер),
который выводит в отладочный журнал фамилию студента, выполнившего
работу. Драйвер должен компилироваться в отдельный файл, а не быть частью
ядра.

\end{flushleft}
\pagebreak

\section{Постановка задачи}
\begin{flushleft}
ЧАСТЬ 1. ДРАЙВЕР В REACTOS \newline
В созданном в лабораторной работе № 1 рабочем дереве операционной
системы ReactOS создать новый модуль, реализующий простейший драйвер,
совместимый с операционными системами Windows NT / ReactOS. Драйвер
должен реализовать минимальный набор функций, необходимый для загрузки и
выгрузки этого драйвера. В функции инициализации этого драйвера DriverEntry() осуществить вывод в отладочный лог используя макрос DPRINT1()
фамилию студента, выполнившего работу. \newline

ЧАСТЬ 2. ДРАЙВЕР В NETBSD \newline
В виртуальной машине с NetBSD, созданной в лабораторной работе № 1,
создать новый загружаемый модуль ядра (loadable kernel module), реализующий
простейший драйвер. Драйвер должен соедржать минимальный набор функций,
необходимый для загрузки и выгрузки этого драйвера. В функции инициализации этого драйвера осуществить вывод в отладочный лог фамилию
студента, выполнившего работу.

\end{flushleft}
\pagebreak

\section{Практическая реализация}
\begin{flushleft}
ЧАСТЬ 1 \newline
В скачанном ранее каталоге с исходным кодом ReactOS[1] в папке reactos/drivers я создал папку lab2\char`_ driver. В drivers/CMakeLists.txt добавил строку add\char`_ subdirectory(lab2\char`_ driver) для того, чтобы мой драйвер участвовал в сборке
системы. В папке drivers/lab2\char`_ driver были созданы три файла: CMakeLists.txt, lab2\char`_ driver.c и lab2\char`_ driver.rc. Содержимое этих файлов было создано на основе уже существующих драйверов. Содержимое CMakeLists.txt:
\end{flushleft}
\begin{minted}[frame=lines,framesep=2mm,baselinestretch=1.2,fontsize=\footnotesize,linenos]{cpp}
add_library(lab2_driver MODULE lab2_driver.c lab2_driver.rc)
set_module_type(lab2_driver kernelmodedriver)
add_importlibs(lab2_driver ntoskrnl)
add_cd_file(TARGET lab2_driver DESTINATION reactos/system32/drivers FOR all)
\end{minted}

\begin{flushleft}
lab2\char`_ driver.c:[2]
\end{flushleft}

\begin{minted}[frame=lines,framesep=2mm,baselinestretch=1.2,fontsize=\footnotesize,linenos]{cpp}
#include <ntddk.h>
#include <debug.h>


static DRIVER_UNLOAD DriverUnload;
static VOID NTAPI
DriverUnload(IN PDRIVER_OBJECT DriverObject)
{
    IoDeleteDevice(DriverObject->DeviceObject);
	DPRINT1("NASHCHEKIN NIKITA'S Driver unloaded\n");
}

NTSTATUS NTAPI DriverEntry(IN PDRIVER_OBJECT DriverObject,
            IN PUNICODE_STRING RegistryPath) {
    /* For non-used parameter */
    UNREFERENCED_PARAMETER(RegistryPath);
    
    /* Setup the Driver Object */
    DriverObject->DriverUnload = DriverUnload;

    DPRINT1("NASHCHEKIN NIKITA!!! From driver\n");
    
    /* Return success. */
    return STATUS_SUCCESS;
}
\end{minted}

\begin{flushleft}
lab2\char`_ driver.rc:
\end{flushleft}

\begin{minted}[frame=lines,framesep=2mm,baselinestretch=1.2,fontsize=\footnotesize,linenos]{cpp}
#define REACTOS_VERSION_DLL
#define REACTOS_STR_FILE_DESCRIPTION  "my driver"
#define REACTOS_STR_INTERNAL_NAME     "lab2_driver"
#define REACTOS_STR_ORIGINAL_FILENAME "lab2_driver.sys"
#include <reactos/version.rc>
\end{minted}

\begin{flushleft}
Далее в среде сборки был пересобран образ, а затем переустановлена система. В работающей системе были выполнены команды:
\end{flushleft}

\begin{minted}[frame=lines,framesep=2mm,baselinestretch=1.2,fontsize=\footnotesize,linenos ,escapeinside=||]{cpp}

sc create lab2driver binPath=C:|\textbackslash|ReactOS|\textbackslash|system32|\textbackslash|drivers|\textbackslash|lab2_driver.sys type= kernel
sc start lab2driver

\end{minted}

\begin{flushleft}
В логи вывелись мои фамилия и имя. \newline
ЧАСТЬ 2 \newline
Сначала был создан файл /usr/src/sys/dev/lab2.c и добавлена реализация драйвера:
\end{flushleft}

\begin{minted}[frame=lines,framesep=2mm,baselinestretch=1.2,fontsize=\footnotesize,linenos]{cpp}
#include <sys/module.h>
MODULE(MODULE_CLASS_MISC, lab2, NULL);

static int lab2_modcmd(modcmd_t cmd, void* arg) {
	printf("NASHCHEKIN NIKITA From NetBSD driver!!!");
	return 0;
}
\end{minted}

\begin{flushleft}
Далее был создан файл /usr/src/sys/modules/lab2/Makefile со следующим содержимым:
\end{flushleft}

\begin{minted}[frame=lines,framesep=2mm,baselinestretch=1.2,fontsize=\footnotesize,linenos ,escapeinside=||]{cpp}
.include "../Makefile.inc"
KMOD=lab2
.PATH: |\textdollar|{S}/dev
SRCS=lab2.c
.include <bsd.kmodule.mk>
\end{minted}

\begin{flushleft}
Затем была выполнена команда make и modload ./lab2.kmod \newline
В логи вывелись мои фамилия и имя.
\end{flushleft}
\pagebreak

\section{Результаты}
\begin{flushleft}
Были реализованы простейшие драйверы, выводящие мои фамилию и имя в лог при их запуске на операционных системах ReactOS и NetBSD. 

\begin{figure}[H]

\centering

\includegraphics[width=0.8\linewidth]{1.png}

\caption{Скриншот 1}

\label{fig:mpr}

\end{figure}

\begin{figure}[H]

\centering

\includegraphics[width=0.8\linewidth]{2.png}

\caption{Скриншот 2}

\label{fig:mpr}

\end{figure}

\begin{figure}[H]

\centering

\includegraphics[width=0.8\linewidth]{3.png}

\caption{Скриншот 3}

\label{fig:mpr}

\end{figure}

\end{flushleft}
\pagebreak 

\section{Выводы}
Выполнив эту лабораторную работу, я разработал два простых драйвера для ReactOS и NetBSD. Создание драйвера для ReactOS было осложнено необходимостью настройки CMakeLists.txt и lab2\char`_ driver.rc, а также тем, что необходимо полностью пересобрать ядро и переустановить систему. На NetBSD можно скомпилировать только драйвер изнутри системы и сразу его запустить без переустановки.
\pagebreak

\section{Список литературы}
\begin{flushleft}
[1] - https://reactos.org/wiki/Building\char`_ ReactOS\newline
[2] - https://reactos.org/forum/viewtopic.php?t=16944\newline
\end{flushleft}
\end{document}
