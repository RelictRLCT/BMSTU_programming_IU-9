\documentclass[a4paper, 14pt]{extarticle}

 
% Поля
%--------------------------------------
\usepackage{minted}
\usepackage{xcolor}
\usepackage{geometry}
\usepackage{float}
\usepackage{ragged2e}
\geometry{a4paper,tmargin=2cm,bmargin=2cm,lmargin=3cm,rmargin=1cm}
%--------------------------------------

% Настройка minted
%--------------------------------------
\definecolor{bg}{rgb}{0.95,0.95,0.95} % Цвет фона
\definecolor{commentgreen}{rgb}{0,0.5,0} % Цвет комментариев

\setminted{
	frame=lines,             % рамка сверху и снизу
	framesep=2mm,            % отступ от рамки до кода
	baselinestretch=1.2,     % межстрочный интервал
	fontsize=\footnotesize,  % размер шрифта
	linenos,                 % нумерация строк
	breaklines,              % перенос строк, если они длинные
	%bgcolor=bg,              % цвет фона
	tabsize=4,               % размер табуляции
	numbersep=5pt,           % отступ номеров строк от кода
	escapeinside=||,         % возможность вставить LaTeX внутрь кода
}
%--------------------------------------

%Russian-specific packages
%--------------------------------------
\usepackage[T2A]{fontenc}
\usepackage[utf8]{inputenc} 
\usepackage[english, main=russian]{babel}
%--------------------------------------

\usepackage{textcomp}

% Красная строка
%--------------------------------------
\usepackage{indentfirst}               
%--------------------------------------             


%Graphics
%--------------------------------------
\usepackage{graphicx}
\graphicspath{ {./images/} }
\usepackage{wrapfig}
%--------------------------------------

% Полуторный интервал
%--------------------------------------
\linespread{1.3}                    
%--------------------------------------

%Выравнивание и переносы
%--------------------------------------
% Избавляемся от переполнений
\sloppy
% Запрещаем разрыв страницы после первой строки абзаца
\clubpenalty=10000
% Запрещаем разрыв страницы после последней строки абзаца
\widowpenalty=10000
%--------------------------------------

%Списки
\usepackage{enumitem}

%Подписи
\usepackage{caption} 

%Гиперссылки
\usepackage{hyperref}

\hypersetup {
	unicode=true
}

%Рисунки
%--------------------------------------
\DeclareCaptionLabelSeparator*{emdash}{~--- }
\captionsetup[figure]{labelsep=emdash,font=onehalfspacing,position=bottom}
%--------------------------------------

\usepackage{newtxtext,newtxmath}
\usepackage{amsmath}
\usepackage{color}
\usepackage{listings}
\lstset{
  belowcaptionskip=1\baselineskip,
  breaklines=true,
  frame=L,
  xleftmargin=\parindent,
  language=C++,
  showstringspaces=false,
  basicstyle=\footnotesize\ttfamily,
  keywordstyle=\bfseries\color{blue},
  commentstyle=\itshape\color{purple},
  identifierstyle=\color{black},
  stringstyle=\color{red},
  extendedchars=\true,
}

%--------------------------------------
%			НАЧАЛО ДОКУМЕНТА
%--------------------------------------

\begin{document}

%--------------------------------------
%			ТИТУЛЬНЫЙ ЛИСТ
%--------------------------------------
\begin{titlepage}
\thispagestyle{empty}
\newpage


%Шапка титульного листа
%--------------------------------------
\vspace*{-60pt}
\hspace{-65pt}
\begin{minipage}{0.3\textwidth}
\hspace*{-20pt}\centering
\includegraphics[width=\textwidth]{emblem.png}
\end{minipage}
\begin{minipage}{0.67\textwidth}\small \textbf{
\vspace*{-0.7ex}
\hspace*{-6pt}\centerline{Министерство науки и высшего образования Российской Федерации}
\vspace*{-0.7ex}
\centerline{Федеральное государственное автономное образовательное учреждение }
\vspace*{-0.7ex}
\centerline{высшего образования}
\vspace*{-0.7ex}
\centerline{<<Московский государственный технический университет}
\vspace*{-0.7ex}
\centerline{имени Н.Э. Баумана}
\vspace*{-0.7ex}
\centerline{(национальный исследовательский университет)>>}
\vspace*{-0.7ex}
\centerline{(МГТУ им. Н.Э. Баумана)}}
\end{minipage}
%--------------------------------------

%Полосы
%--------------------------------------
\vspace{-25pt}
\hspace{-35pt}\rule{\textwidth}{2.3pt}

\vspace*{-20.3pt}
\hspace{-35pt}\rule{\textwidth}{0.4pt}
%--------------------------------------

\vspace{1.5ex}
\hspace{-35pt} \noindent \small ФАКУЛЬТЕТ\hspace{80pt} <<Информатика и системы управления>>

\vspace*{-16pt}
\hspace{47pt}\rule{0.83\textwidth}{0.4pt}

\vspace{0.5ex}
\hspace{-35pt} \noindent \small КАФЕДРА\hspace{50pt} <<Теоретическая информатика и компьютерные технологии>>

\vspace*{-16pt}
\hspace{30pt}\rule{0.866\textwidth}{0.4pt}
  
\vspace{11em}

\begin{center}
\Large {\bf Лабораторная работа № 5} \\ 
\large {\bf по курсу <<Численные методы>>} \\ 
\end{center}\normalsize

\vspace{8em}


\begin{flushright}
  {Студент группы ИУ9-62Б Нащёкин Н.Д.\hspace*{15pt} \\
  \vspace{2ex}
  Преподаватель: Домрачева А.Б.\hspace*{15pt}}
\end{flushright}

\bigskip

\vfill
 

\begin{center}
\textsl{Москва, 2025}
\end{center}
\end{titlepage}
%--------------------------------------
%		КОНЕЦ ТИТУЛЬНОГО ЛИСТА
%--------------------------------------

\renewcommand{\ttdefault}{pcr}

\setlength{\tabcolsep}{3pt}
\newpage
\setcounter{page}{2}

\section{Задача}
\begin{justify}
1. Найти минимум функции двух переменных с точностью $\varepsilon = 0.001$, начиная 
итерации из точки $X^0$. \\
2. Найти минимум аналитически.\\
3. Сравнить полученные результаты.  


\end{justify}
\pagebreak

\section{Основная теория}
\begin{justify}

\noindent\large {Метод наискорейшего спуска}\\ \normalsize

Метод наискорейшего спуска -- итерационный метод, позволяющий найти 
точку минимума заданной функции. Пусть для функции $f(x_1, ..., x_n)$ на $k$-ом шаге 
имеем приближение к минимуму $X^k \approx (x^k_1, ..., x^k_n)$. Рассмотрим функцию 
$\varphi_k(t) = f(x^k_1 - t \frac{\partial f}{\partial x_1}(X^k), ..., 
x^k_n - t \frac{\partial f}{\partial x_n}(X^k)) = f(X^k - t*grad(f(X^k)))$. 

Функция $\varphi_k(t)$ -- это ограничение функции $f(X)$ на прямую градиентного 
(наискорейшего) спуска, проходящую через точку $k$-го приближения $X^k$.
Пусть $t^*$ -- точка минимума этой функции. Тогда следующее приближение к 
точке минимума:
$X^{k+1} = X^k -t^* * grad(f(X^k)) = x^k_1 - t^* \frac{\partial f}{\partial x_1}(X^k), ..., x^k_n - t^* \frac{\partial f}{\partial x_n}(X^k)$.

Итерации продолжаются, пока не будет выполнено условие завершения счёта: 
\[	
	||grad(f(X^k))|| = \max_{1 \le i \le n}{|\frac{\partial f}{\partial x_i}(X^k)|} < \varepsilon
\]

Чаще всего точно искать минимум функции $\varphi(t)$ не нужно и достаточно 
ограничиться лишь одним приближением и поиском $t^k$, например, по методу парабол.
Тогда в двумерном случае приближения будут особенно простыми: 
\[
	(x_{k+1}, y_{k+1}) = (x_k -t^k \frac{\partial f}{\partial x}, 
	y_k -t^k \frac{\partial f}{\partial y}).
\]
Здесь $t^k = -\frac{\varphi'_k(0)}{\varphi''_k(0)}$, где 
$\varphi'_k(0) = - (\frac{\partial f}{\partial x})^2 
- (\frac{\partial f}{\partial y})^2$, 
$\varphi''_k(0) = \frac{\partial^2 f}{\partial x^2} (\frac{\partial f}{\partial x})^2  + 2 \frac{\partial^2 f}{\partial x \partial y} \frac{\partial f}{\partial x} 
\frac{\partial f}{\partial y} + \frac{\partial^2 f}{\partial y^2} 
(\frac{\partial f}{\partial y})^2$, все частные производные здесь берутся в точке 
$(x_k, y_k)$.


\end{justify}
\pagebreak

\section{Практическая реализация}
Листинг 1 — реализация программы
\begin{minted}{python}
from sympy import *


# f(x) = 2 * x_1^2 + 3 * x_2^2 - 2 * sin((x_1 - x_2) / 2) + x_2
# x0 = (0, 0)


eps = 0.001


x_1 = Symbol("x_1")
x_2 = Symbol("x_2")
f = 2 * x_1 ** 2 + 3 * x_2 ** 2 - 2 * sin((x_1 - x_2) / 2) + x_2


def norm(point: dict):
	df_x1 = diff(f, x_1).subs(point).evalf()
	df_x2 = diff(f, x_2).subs(point).evalf()
	return max(abs(df_x1), abs(df_x2))


def phi_k_first(x_k):
	return -(diff(f, x_1).subs(x_k) ** 2).evalf() - (diff(f, x_2).subs(x_k) ** 2).evalf()


def phi_k_second(x_k):
	return (
		diff(f, x_1, 2).subs(x_k).evalf() * diff(f, x_1).subs(x_k).evalf() ** 2
		+ 2
		* diff(f, x_1, x_2).subs(x_k).evalf()
		* diff(f, x_1).subs(x_k).evalf()
		* diff(f, x_2).subs(x_k).evalf()
		+ diff(f, x_2, 2).subs(x_k).evalf() * diff(f, x_2).subs(x_k).evalf() ** 2
	)


def t_k(x_k):
	return - phi_k_first(x_k) / phi_k_second(x_k)


def grad_down():
	x_k = {x_1: 0, x_2: 0}
	x_k1 = {x_1: 0, x_2: 0}
	while norm(x_k) >= eps:
		x_k = x_k1
		x_k1 = {
			x_1: x_k[x_1] - t_k(x_k) * diff(f, x_1).subs(x_k).evalf(),
			x_2: x_k[x_2] - t_k(x_k) * diff(f, x_2).subs(x_k).evalf()
		}
	return x_k1


def find_min():
	df_dx1 = diff(f, x_1)
	df_dx2 = diff(f, x_2)
	critical_point = nsolve([df_dx1, df_dx2], (x_1, x_2), [0, 0])
	return critical_point


def main():
	res_numeric = grad_down()
	res_analytic = find_min()
	
	print(f'eps = {eps}')
	print(f'Точка минимума: (x, y) = ({res_numeric[x_1]:.6f}, {res_numeric[x_2]:.6f})')
	print(f'Точка минимума, вычисленная аналитически: (x, y) = ({res_analytic[0]:.6f}, {res_analytic[1]:.6f})')
	print(f'Абсолютные погрешности. По x: {abs(res_numeric[x_1] - res_analytic[0]):.6f}; '
		f'по y: {abs(res_numeric[x_2] - res_analytic[1]):.6f}')
	print(f"Норма в найденной точке минимума: {norm(res_numeric):.6f}")


if __name__ == "__main__":
main()


\end{minted}

Мой вариант -- 22. По условию требуется найти минимум функции $f = 2 x_1^2 + 3 x_2^2 - 2 sin(\frac{x_1 - x_2}{2}) + x_2$, $X^0 = (0, 0)$. 
Метод наискорейшего спуска реализован 
в функции $grad\_down()$. Также по условию необходимо сравнить полученный результат с аналитическим решением. Для этого, например с использованием библиотеки символьных 
вычислений $SymPy$, нужно найти решение системы 
\[
\begin{cases}
	\frac{\partial f}{\partial x} = 0 \\
	\frac{\partial f}{\partial y} = 0.
\end{cases}
\]
Решения этой системы -- критические точки, которые нужно проверить на наличие 
экстремума с помощью достаточного условия его существования: 
$\frac{\partial^2 f}{\partial x^2} \frac{\partial^2 f}{\partial y^2} - \frac{\partial^2 f}{\partial x \partial y} > 0$, причём если $\frac{\partial^2 f}{\partial x^2} > 0$, то это точка минимума (нас интересует этот случай). Но в случае 
с заданной функцией критическая точка только одна, она и является точкой минимума, 
поэтому дополнительную проверку достаточного условия можно не проводить.  
Результат работы программы приведён на рисунке 1.


\begin{figure}[H]
	
	\centering
	
	\includegraphics[width=0.8\linewidth]{1.png}
	\captionsetup{justification=centering}
	\caption{Результат выполнения программы}
	
	\label{fig:mpr}
	
\end{figure}

\pagebreak

\section{Вывод}
В данной лабораторной работе был реализован метод наискорейшего спуска для 
поиска точки минимума функции двух переменных. Кроме того, требуемая точка также 
была найдена аналитически. Результаты численного и аналитического поиска, а также норма градиента заданной функции в численно найденной точке выводятся на экран. 
\end{document}
