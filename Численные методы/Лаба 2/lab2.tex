\documentclass[a4paper, 14pt]{extarticle}

 
% Поля
%--------------------------------------
\usepackage{minted}
\usepackage{xcolor}
\usepackage{geometry}
\usepackage{float}
\usepackage{ragged2e}
\geometry{a4paper,tmargin=2cm,bmargin=2cm,lmargin=3cm,rmargin=1cm}
%--------------------------------------

% Настройка minted
%--------------------------------------
\definecolor{bg}{rgb}{0.95,0.95,0.95} % Цвет фона
\definecolor{commentgreen}{rgb}{0,0.5,0} % Цвет комментариев

\setminted{
	frame=lines,             % рамка сверху и снизу
	framesep=2mm,            % отступ от рамки до кода
	baselinestretch=1.2,     % межстрочный интервал
	fontsize=\footnotesize,  % размер шрифта
	linenos,                 % нумерация строк
	breaklines,              % перенос строк, если они длинные
	%bgcolor=bg,              % цвет фона
	tabsize=4,               % размер табуляции
	numbersep=5pt,           % отступ номеров строк от кода
	escapeinside=||,         % возможность вставить LaTeX внутрь кода
}
%--------------------------------------

%Russian-specific packages
%--------------------------------------
\usepackage[T2A]{fontenc}
\usepackage[utf8]{inputenc} 
\usepackage[english, main=russian]{babel}
%--------------------------------------

\usepackage{textcomp}

% Красная строка
%--------------------------------------
\usepackage{indentfirst}               
%--------------------------------------             


%Graphics
%--------------------------------------
\usepackage{graphicx}
\graphicspath{ {./images/} }
\usepackage{wrapfig}
%--------------------------------------

% Полуторный интервал
%--------------------------------------
\linespread{1.3}                    
%--------------------------------------

%Выравнивание и переносы
%--------------------------------------
% Избавляемся от переполнений
\sloppy
% Запрещаем разрыв страницы после первой строки абзаца
\clubpenalty=10000
% Запрещаем разрыв страницы после последней строки абзаца
\widowpenalty=10000
%--------------------------------------

%Списки
\usepackage{enumitem}

%Подписи
\usepackage{caption} 

%Гиперссылки
\usepackage{hyperref}

\hypersetup {
	unicode=true
}

%Рисунки
%--------------------------------------
\DeclareCaptionLabelSeparator*{emdash}{~--- }
\captionsetup[figure]{labelsep=emdash,font=onehalfspacing,position=bottom}
%--------------------------------------

\usepackage{newtxtext,newtxmath}
\usepackage{amsmath}
\usepackage{color}
\usepackage{listings}
\lstset{
  belowcaptionskip=1\baselineskip,
  breaklines=true,
  frame=L,
  xleftmargin=\parindent,
  language=C++,
  showstringspaces=false,
  basicstyle=\footnotesize\ttfamily,
  keywordstyle=\bfseries\color{blue},
  commentstyle=\itshape\color{purple},
  identifierstyle=\color{black},
  stringstyle=\color{red},
  extendedchars=\true,
}

%--------------------------------------
%			НАЧАЛО ДОКУМЕНТА
%--------------------------------------

\begin{document}

%--------------------------------------
%			ТИТУЛЬНЫЙ ЛИСТ
%--------------------------------------
\begin{titlepage}
\thispagestyle{empty}
\newpage


%Шапка титульного листа
%--------------------------------------
\vspace*{-60pt}
\hspace{-65pt}
\begin{minipage}{0.3\textwidth}
\hspace*{-20pt}\centering
\includegraphics[width=\textwidth]{emblem.png}
\end{minipage}
\begin{minipage}{0.67\textwidth}\small \textbf{
\vspace*{-0.7ex}
\hspace*{-6pt}\centerline{Министерство науки и высшего образования Российской Федерации}
\vspace*{-0.7ex}
\centerline{Федеральное государственное автономное образовательное учреждение }
\vspace*{-0.7ex}
\centerline{высшего образования}
\vspace*{-0.7ex}
\centerline{<<Московский государственный технический университет}
\vspace*{-0.7ex}
\centerline{имени Н.Э. Баумана}
\vspace*{-0.7ex}
\centerline{(национальный исследовательский университет)>>}
\vspace*{-0.7ex}
\centerline{(МГТУ им. Н.Э. Баумана)}}
\end{minipage}
%--------------------------------------

%Полосы
%--------------------------------------
\vspace{-25pt}
\hspace{-35pt}\rule{\textwidth}{2.3pt}

\vspace*{-20.3pt}
\hspace{-35pt}\rule{\textwidth}{0.4pt}
%--------------------------------------

\vspace{1.5ex}
\hspace{-35pt} \noindent \small ФАКУЛЬТЕТ\hspace{80pt} <<Информатика и системы управления>>

\vspace*{-16pt}
\hspace{47pt}\rule{0.83\textwidth}{0.4pt}

\vspace{0.5ex}
\hspace{-35pt} \noindent \small КАФЕДРА\hspace{50pt} <<Теоретическая информатика и компьютерные технологии>>

\vspace*{-16pt}
\hspace{30pt}\rule{0.866\textwidth}{0.4pt}
  
\vspace{11em}

\begin{center}
\Large {\bf Лабораторная работа № 2} \\ 
\large {\bf по курсу <<Численные методы>>} \\ 
\end{center}\normalsize

\vspace{8em}


\begin{flushright}
  {Студент группы ИУ9-62Б Нащёкин Н.Д.\hspace*{15pt} \\
  \vspace{2ex}
  Преподаватель: Домрачева А.Б.\hspace*{15pt}}
\end{flushright}

\bigskip

\vfill
 

\begin{center}
\textsl{Москва, 2025}
\end{center}
\end{titlepage}
%--------------------------------------
%		КОНЕЦ ТИТУЛЬНОГО ЛИСТА
%--------------------------------------

\renewcommand{\ttdefault}{pcr}

\setlength{\tabcolsep}{3pt}
\newpage
\setcounter{page}{2}

\section{Задача}
\begin{justify}
• Найти $\int_a^b f(x)\,dx$ по формуле трапеций и по формуле Симпсона с погрешностью $\varepsilon = 0.001$. Сравнить
требуемое число разбиений для обоих методов.


\end{justify}
\pagebreak

\section{Основная теория}
\begin{justify}

\noindent\large {Метод средних прямоугольников}\\ \normalsize

Пусть функция $f(x)$ непрерывна на отрезке $[a, b]$. Разобьём отрезок на $n$ равных отрезков точками
$a = x_0, x_1, ..., x_n = b$. Тогда фигура под графиком функции $f(x)$ разобьётся на криволинейные 
трапеции на полученных отрезках. Пусть $h = x_{i} - x_{i-1} = \frac{b - a}{n}, i = 1, ..., n$ -- постоянный шаг разбиения. 

Для начала рассмотрим так называемые методы левых и правых прямоугольников. На каждом частичном отрезке возьмём левую границу и посчитаем площадь прямоугольника со сторонами 
$f(x_{i-1})$ и $(x_{i} - x_{i-1}) = h$. Тогда $I_{li} = hf(x_{i-1})$ -- площади левых прямоугольников, 
$i = 1, ..., n$. В таком случае $I_{l} = h \sum_{i=1}^{n} f(x_{i-1})$ -- приближённое методом левых прямоугольников 
значение искомого интеграла. 

Аналогично, взяв вместо левых границ частичных отрезков правые, получим 
$I_{ri} = hf(x_{i})$ -- площади правых прямоугольников, $I_{r} = h \sum_{i=1}^{n} f(x_{i})$ -- приближённое значение
интеграла. 

Недостаток этих методов в том, что они имеют первый порядок точности. Существует метод средних (центральных) прямоугольников, 
имеющий второй порядок точности. Для его реализации вместо левой или правой границы частичного отрезка берётся средняя точка: 
$I_{si} = hf(x_{i}-\frac{h}{2})$. В таком случае приближённое значение интеграла: $I_{s} = h \sum_{i=1}^{n} f(x_{i} - \frac{h}{2})$. 

Выведем погрешность метода средних прямоугольников. Приближённая формула для площади одного сегмента:
\[
\int_{x_{i-1}}^{x_i} f(x)\, dx \approx f\left(x_{i-1} + \frac{h}{2} \right) h,
\]
где \( h = x_i - x_{i-1} \).

Погрешность на одном сегменте:
\[
\delta_i = \int_{x_{i-1}}^{x_i} f(x)\, dx - f\left(x_{i-1} + \frac{h}{2} \right) h.
\]

Разложим функцию \( f(x) \) в ряд Тейлора в точке \( x_{i-1} + \frac{h}{2} \):

\[
f(x) = f\left(x_{i-1} + \frac{h}{2}\right) + f'\left(x_{i-1} + \frac{h}{2}\right)\left(x - \left(x_{i-1} + \frac{h}{2}\right)\right) + \frac{f''(\varepsilon_i)}{2}\left(x - \left(x_{i-1} + \frac{h}{2}\right)\right)^2,
\]
где \( \varepsilon_i \in [x_{i-1}, x_i] \).

Подставляя в выражение для \( \delta_i \), получаем:

\[
\delta_i = \int_{x_{i-1}}^{x_i} \left(f(x) - f\left(x_{i-1} + \frac{h}{2}\right)\right) dx = \frac{f''(\varepsilon_i) h^3}{24}, 
\]

Абсолютная погрешность на одном отрезке:

\[
|\delta_i| \leq \max_{x \in [x_{i-1}, x_i]} |f''(x)| \frac{h^3}{24}.
\]

Для всей фигуры (всего $n$ равных сегментов, $h = \frac{b-a}{n}$):

\[
|\delta| \leq \max_{x \in [a, b]} |f''(x)| \frac{(b - a)}{24}h^2 = \mathcal{O}(h^2).
\]

\noindent\large {Метод трапеций}\\ \normalsize

Теперь в предыдущих обозначениях рассмотрим похожий метод приближённого нахождения определённого интеграла. Как и ранее,
фигура под графиком функции $f(x)$ разбита на множество криволинейных трапеций.
Тогда площадь каждой криволинейной трапеции примерно равна 
\[
\frac{f(x_{i-1}) + f(x_i)}{2}h. 
\]
Суммируя полученные
площади, получим примерное значение интеграла $I = \int_a^b f(x)\,dx$:
\[ 
I \approx \frac{1}{2}h(f(x_0) + f(x_1) + f(x_1) + 
f(x_2) + f(x_2) + ... + f(x_{n-1}) + f(x_{n-1}) + f(x_n)). 
\]
Заметим, что в полученной формуле все $f(x_i)$,
 кроме $f(x_0) = f(a)$ и $f(x_n) = f(b)$, встречаются дважды. Учитывая это, получим: 
\[
I \approx h(\frac{f(a) + f(b)}{2} + \sum_{i=1}^{n-1} f(x_i)). \\
\]

Метод трапеций имеет второй порядок точности, как и метод средних прямоугольников.
Аналогично нахождению погрешности для метода средних прямоугольников, можно найти погрешность для метода трапеций:
\[
|\delta| \leq \max_{x \in [a, b]} |f''(x)| \frac{(b - a)}{12}h^2 = \mathcal{O}(h^2).
\]
Как видно, для метода трапеций оценка погрешности в два раза хуже. \\

\noindent\large {Метод Симпсона}\\ \normalsize

В предыдущих обозначениях также верна приближённая формула: 
\[
I \approx \frac{h}{3}(f(x_0) + f(x_n) + 4(f(x_1) + f(x_3) + ... + f(x_{n-1}) + 2(f(x_2) + f(x_4) + ... + f(x_{n-2})))).
\]
Здесь число отрезков $n$ должно быть чётным.

Получим оценку погрешности для метода Симпсона: 
\[
\int_{x_{i-1}}^{x_{i+1}} f(x) \, dx \approx \frac{h}{3} \left(f(x_{i-1}) + 4f(x_i) + f(x_{i+1}) \right),
\]
где $h = \frac{x_{i+1} - x_{i-1}}{2}$ — половина ширины параболического интервала.

Для функции $f \in C^4[a, b]$ погрешность на таком двойном отрезке можно выразить как:

\[
\delta_i = -\frac{h^5}{90} f^{(4)}(\xi_i), \quad \xi_i \in [x_{i-1}, x_{i+1}].
\]

Если отрезок $[a, b]$ разбит на $n$ равных частей (где $n$ чётное), и $h = \frac{b - a}{n}$, то суммарная погрешность метода Симпсона на всём отрезке оценивается как:

\[
\left| \int_a^b f(x)\, dx - I \right| \leq \max_{x \in [a, b]} \left| f^{(4)}(x) \right| \frac{(b - a)^5}{180 n^4},
\]
где $I$ — приближённое значение интеграла, полученное по формуле Симпсона. Учитывая, что $h = \frac{b - a}{n}$, получим
\[
|\delta| \leq \max_{x \in [a, b]} \left| f^{(4)}(x) \right| \frac{(b - a)}{180}h^4 = \mathcal{O}(h^4).
\]

\noindent\large {Правило Рунге}\\ \normalsize

Для оценки погрешности на практике часто применяют правило Рунге: необходимо найти приближённое значение $S$ с шагом 
$h$ и с шагом $2h$, а затем посчитать $\Delta S(h) = \frac{S(h) - S(2h)}{2^p - 1}$, где $p = 2$ для метода трапеций и 
метода прямоугольников и $p = 4$ для метода Симпсона. Теперь, когда при вычислениях с последовательным 
увеличением $n$ достигается неравенство $|\Delta S(h)| \le \varepsilon$, где $\varepsilon$ -- некоторая 
заданная абсолютная погрешность, вычисления прекращаются.

\end{justify}
\pagebreak

\section{Практическая реализация}
Листинг 1 — реализация программы
\begin{minted}{python}
import math
# Интеграл = 8.0366775 - Вычислено калькулятором


def f(x: float) -> float:
	return math.log(2 * x)


eps = 0.001


def trap_method(a: float, b: float, n: int) -> float:
	h = (b - a) / n
	summ = 0
	for i in range(1, n, 1):
		summ += f(a + i * h)
	summ += (f(a) + f(b)) / 2
	return summ * h


def simpson_method(a: float, b: float, n: int) -> float:
	h = (b - a) / n
	sum1 = 0
	sum2 = 0
	for i in range(1, n, 2):
		sum1 += f(a + i * h)
	
	for i in range(2, n, 2):
		sum2 += f(a + i * h)
	
	summ = 4*sum1 + 2*sum2 + f(a) + f(b)
	return summ * h / 3


def average_rectangle(a: float, b: float, n: int) -> float:
	h = (b - a) / n
	summ = 0
	for i in range(0, n, 1):
		summ += f(a + i * h + h / 2)
	return summ * h


def runge_rule(S_h, S_2h, p):
	return (S_h - S_2h) / (2 ** p - 1)


def main():
	a = 0.5
	b = 2 * math.e
	
	n_t = 2
	S_2h = trap_method(a, b, n_t)
	S_h = trap_method(a, b, 2 * n_t)
	S_delta = runge_rule(S_h, S_2h, p=2)
	while abs(S_delta) > eps:
		n_t *= 2
		S_2h = trap_method(a, b, n_t)
		S_h = trap_method(a, b, 2 * n_t)
		S_delta = runge_rule(S_h, S_2h, p=2)
	
	n_s = 2
	S_2h_s = simpson_method(a, b, n_s)
	S_h_s = simpson_method(a, b, 2 * n_s)
	S_delta_s = runge_rule(S_h_s, S_2h_s, p=4)
	while abs(S_delta_s) > eps:
		n_s *= 2
		S_2h_s = simpson_method(a, b, n_s)
		S_h_s = simpson_method(a, b, 2 * n_s)
		S_delta_s = runge_rule(S_h_s, S_2h_s, p=4)
	
	n_s_t = 2
	S_2h_s_t = average_rectangle(a, b, n_s_t)
	S_h_s_t = average_rectangle(a, b, 2 * n_s_t)
	S_delta_s_t = runge_rule(S_h_s_t, S_2h_s_t, p=2)
	while abs(S_delta_s_t) > eps:
		n_s_t *= 2
		S_2h_s_t = average_rectangle(a, b, n_s_t)
		S_h_s_t = average_rectangle(a, b, 2 * n_s_t)
		S_delta_s_t = runge_rule(S_h_s_t, S_2h_s_t, p=2)
	
	print(f'     Метод трапеций    Метод Симпсона   Метод сред. прямоугольников')
	print(f'n          {n_t}                {n_s}                    {n_s_t}')
	print(f'I*         {S_h:.6f}          {S_h_s:.6f}          {S_h_s_t:.6f}')
	print(f'R          {S_delta:.6f}          {S_delta_s:.6f}         {S_delta_s_t:.6f}')
	print(f'I* + R     {S_h + S_delta:.6f}          {S_h_s + S_delta_s:.6f}          {S_h_s_t + S_delta_s_t:.6f}')


if __name__ == "__main__":
	main()

\end{minted}

Мой вариант -- 22. По условию необходимо найти приближённое значение интеграла $\int_{0.5}^{2e} ln(2x)\,dx$.
Все три метода вычислений реализованы в виде отдельных функций. Также для удобства была выделена 
функция для вычисления $\Delta S(h)$ по правилу Рунге. На рисунке 1 представлен результат работы программы. Здесь $n$ -- 
число разбиений, $I*$ -- вычисленное приближённое значение интеграла, $R$ -- последняя $\Delta S(h)$ (уточнение по 
Ричардсону).

\begin{figure}[H]
	
	\centering
	
	\includegraphics[width=0.8\linewidth]{1.png}
	\captionsetup{justification=centering}
	\caption{Результат выполнения программы}
	
	\label{fig:mpr}
	
\end{figure}

\pagebreak

\section{Вывод}
В данной лабораторной работе были реализованы метод трапеций, метод средних прямоугольников и метод Симпсона 
для приближённого вычисления значения определённого интеграла. Из выведенных программой результатов видно, 
что метод Симпсона позволяет найти значение интеграла с заданной погрешностью с гораздо меньшим числом разбиений: для метода трапеций и средних прямоугольников это значение равно $32$, а для метода Симпсона -- $8$. Кроме того, оба метода 
корректно вычисляют приближённое значение заданного интеграла: значение, полученное на калькуляторе, равно $8.0366775$.
\end{document}
