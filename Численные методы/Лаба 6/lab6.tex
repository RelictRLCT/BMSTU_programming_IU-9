\documentclass[a4paper, 14pt]{extarticle}

 
% Поля
%--------------------------------------
\usepackage{minted}
\usepackage{xcolor}
\usepackage{geometry}
\usepackage{float}
\usepackage{ragged2e}
\geometry{a4paper,tmargin=2cm,bmargin=2cm,lmargin=3cm,rmargin=1cm}
%--------------------------------------

% Настройка minted
%--------------------------------------
\definecolor{bg}{rgb}{0.95,0.95,0.95} % Цвет фона
\definecolor{commentgreen}{rgb}{0,0.5,0} % Цвет комментариев

\setminted{
	frame=lines,             % рамка сверху и снизу
	framesep=2mm,            % отступ от рамки до кода
	baselinestretch=1.2,     % межстрочный интервал
	fontsize=\footnotesize,  % размер шрифта
	linenos,                 % нумерация строк
	breaklines,              % перенос строк, если они длинные
	%bgcolor=bg,              % цвет фона
	tabsize=4,               % размер табуляции
	numbersep=5pt,           % отступ номеров строк от кода
	escapeinside=||,         % возможность вставить LaTeX внутрь кода
}
%--------------------------------------

%Russian-specific packages
%--------------------------------------
\usepackage[T2A]{fontenc}
\usepackage[utf8]{inputenc} 
\usepackage[english, main=russian]{babel}
%--------------------------------------

\usepackage{textcomp}

% Красная строка
%--------------------------------------
\usepackage{indentfirst}               
%--------------------------------------             


%Graphics
%--------------------------------------
\usepackage{graphicx}
\graphicspath{ {./images/} }
\usepackage{wrapfig}
%--------------------------------------

% Полуторный интервал
%--------------------------------------
\linespread{1.3}                    
%--------------------------------------

%Выравнивание и переносы
%--------------------------------------
% Избавляемся от переполнений
\sloppy
% Запрещаем разрыв страницы после первой строки абзаца
\clubpenalty=10000
% Запрещаем разрыв страницы после последней строки абзаца
\widowpenalty=10000
%--------------------------------------

%Списки
\usepackage{enumitem}

%Подписи
\usepackage{caption} 

%Гиперссылки
\usepackage{hyperref}

\hypersetup {
	unicode=true
}

%Рисунки
%--------------------------------------
\DeclareCaptionLabelSeparator*{emdash}{~--- }
\captionsetup[figure]{labelsep=emdash,font=onehalfspacing,position=bottom}
\captionsetup[table]{labelsep=emdash}
%--------------------------------------

\usepackage{newtxtext,newtxmath}
\usepackage{amsmath}
\usepackage{color}
\usepackage{listings}
\lstset{
  belowcaptionskip=1\baselineskip,
  breaklines=true,
  frame=L,
  xleftmargin=\parindent,
  language=C++,
  showstringspaces=false,
  basicstyle=\footnotesize\ttfamily,
  keywordstyle=\bfseries\color{blue},
  commentstyle=\itshape\color{purple},
  identifierstyle=\color{black},
  stringstyle=\color{red},
  extendedchars=\true,
}

%--------------------------------------
%			НАЧАЛО ДОКУМЕНТА
%--------------------------------------

\begin{document}

%--------------------------------------
%			ТИТУЛЬНЫЙ ЛИСТ
%--------------------------------------
\begin{titlepage}
\thispagestyle{empty}
\newpage


%Шапка титульного листа
%--------------------------------------
\vspace*{-60pt}
\hspace{-65pt}
\begin{minipage}{0.3\textwidth}
\hspace*{-20pt}\centering
\includegraphics[width=\textwidth]{emblem.png}
\end{minipage}
\begin{minipage}{0.67\textwidth}\small \textbf{
\vspace*{-0.7ex}
\hspace*{-6pt}\centerline{Министерство науки и высшего образования Российской Федерации}
\vspace*{-0.7ex}
\centerline{Федеральное государственное автономное образовательное учреждение }
\vspace*{-0.7ex}
\centerline{высшего образования}
\vspace*{-0.7ex}
\centerline{<<Московский государственный технический университет}
\vspace*{-0.7ex}
\centerline{имени Н.Э. Баумана}
\vspace*{-0.7ex}
\centerline{(национальный исследовательский университет)>>}
\vspace*{-0.7ex}
\centerline{(МГТУ им. Н.Э. Баумана)}}
\end{minipage}
%--------------------------------------

%Полосы
%--------------------------------------
\vspace{-25pt}
\hspace{-35pt}\rule{\textwidth}{2.3pt}

\vspace*{-20.3pt}
\hspace{-35pt}\rule{\textwidth}{0.4pt}
%--------------------------------------

\vspace{1.5ex}
\hspace{-35pt} \noindent \small ФАКУЛЬТЕТ\hspace{80pt} <<Информатика и системы управления>>

\vspace*{-16pt}
\hspace{47pt}\rule{0.83\textwidth}{0.4pt}

\vspace{0.5ex}
\hspace{-35pt} \noindent \small КАФЕДРА\hspace{50pt} <<Теоретическая информатика и компьютерные технологии>>

\vspace*{-16pt}
\hspace{30pt}\rule{0.866\textwidth}{0.4pt}
  
\vspace{11em}

\begin{center}
\Large {\bf Лабораторная работа № 6} \\ 
\large {\bf по курсу <<Численные методы>>} \\ 
\end{center}\normalsize

\vspace{8em}


\begin{flushright}
  {Студент группы ИУ9-62Б Нащёкин Н.Д.\hspace*{15pt} \\
  \vspace{2ex}
  Преподаватель: Домрачева А.Б.\hspace*{15pt}}
\end{flushright}

\bigskip

\vfill
 

\begin{center}
\textsl{Москва, 2025}
\end{center}
\end{titlepage}
%--------------------------------------
%		КОНЕЦ ТИТУЛЬНОГО ЛИСТА
%--------------------------------------

\renewcommand{\ttdefault}{pcr}

\setlength{\tabcolsep}{3pt}
\newpage
\setcounter{page}{2}

\section{Задача}
\begin{justify}
1. Построить графики таблично заданной функции и функции $z(x)$. \\
2. Найти значения $x_a$, $x_g$, $x_h$, $y_a$, $y_g$, $y_h$, $z(x_a)$, $z(x_g)$, $z(x_h)$, 
$\delta_1$, ..., $\delta_9$, $\delta_k = min(\delta_i)$.\\
3. Составить систему уравнений для определения $a$ и $b$ и решить её.  \\ 
4. Найти среднеквадратичное отклонение $\Delta$.


\end{justify}
\pagebreak

\section{Индивидуальный вариант}
\begin{justify}
	Вариант 22. \\
	Функция задана таблично: \\
	\begin{table}[h!]
		\centering
		\begin{tabular}{|c|ccccccccc|}
			\hline
			$x$ & 1.0 & 1.5 & 2.0 & 2.5 & 3.0 & 3.5 & 4.0 & 4.5 & 5.0 \\
			\hline
			$y$ & 1.24 & 1.74 & 1.61 & 2.16 & 3.06 & 2.88 & 4.53 & 5.40 & 7.07 \\
			\hline
		\end{tabular}
		\caption{Значения $x$ и $y$}
	\end{table}
	
	
\end{justify}
\pagebreak

\section{Основная теория}
\begin{justify}

\noindent\large {Аппроксимация методом наименьших квадратов для двупараметрических моделей}\\ \normalsize

Существует формальный метод выбора вида аппроксимирующей функции, зависящей от двух параметров. 

Введём обозначения: \\ 
$x_a = \frac{x_0 + x_n}{2}$ -- среднее арифметическое двух чисел, \\ 
$x_g = \sqrt{x_0 x_n}$ -- среднее геометрическое двух чисел, \\
$x_h = \frac{2}{\frac{1}{x_0} + \frac{1}{x_n}}$ -- среднее гармоническое двух чисел. \\

Рассмотрим девять видов эмпирических зависимостей, а также их свойства: \\
$z_1(x) = ax + b \iff z(x_a) = z_a$ \\
$z_2(x) = ax^b \iff z(x_g) = z_g$ \\
$z_3(x) = ae^{bx} \iff z(x_a) = z_g$ \\
$z_4(x) = a ln(x) + b \iff z(x_g) = z_a$ \\
$z_5(x) = \frac{a}{x} + b \iff z(x_h) = z_a$ \\
$z_6(x) = \frac{1}{ax + b} \iff z(x_a) = z_h$ \\
$z_7(x) = \frac{x}{ax + b} \iff z(x_h) = z_h$ \\
$z_8(x) = ae^{\frac{b}{x}} \iff z(x_h) = z_g$ \\
$z_9(x) = \frac{1}{a ln(x) + b} \iff z(x_g) = z_h$. \\
Здесь $z_a$, $z_g$, $z_h$ -- среднее арифметическое, среднее геометрическое 
и среднее гармоническое значения функции $z(x)$ соответственно в точках $x_0$ и $x_n$.

Таким образом, для выбора нужной функции из перечисленного набора нужно выполнить 
следующие шаги: \\
1. Нанести на график заданные точки $(x_i, y_i)$ и провести гладкую монотонную кривую $z(x)$, 
аппроксимирующую эту зависимость. \\
2. Вычислить значения $x_a$, $x_g$, $x_h$, $y_a$, $y_g$, $y_h$ относительно $x_0$, $x_n$ и $y_0$, $y_n$ из заданной 
таблицы, а также по построенному графику $z(x)$ определить значения $z(x_a)$, $z(x_g)$ и $z(x_h)$. \\
3. Найти минимальное из нижеперечисленных величин: \\
$\delta_1 = |z(x_a) - y_a|$ , $\delta_2 = |z(x_g) - y_g|$, $\delta_3 = |z(x_a) - y_g|$, \\
$\delta_4 = |z(x_g) - y_a|$ , $\delta_5 = |z(x_h) - y_a|$, $\delta_6 = |z(x_a) - y_h|$, \\
$\delta_7 = |z(x_h) - y_h|$ , $\delta_8 = |z(x_h) - y_g|$, $\delta_9 = |z(x_g) - y_h|$. \\
Номер минимальной $\delta_i$ определяет номер аппроксимирующей функции. Пусть такой номер -- $k$.

Затем для выбранной функции $z_k(x)$ нужно методом наименьших квадратов определить коэффициенты $a$ и $b$. 
Если, например, мы выбрали функцию $z_1(x) = ax + b$, то среднеквадратичное уклонение равно 
$\sum_{i=0}^n \left( a x_i + b - y_i \right)^2$. Коэффициенты $a$ и $b$ находим методом наименьших квадратов: \\
\[
a \sum_{i=0}^n x_i^2 + b \sum_{i=0}^n x_i = \sum_{i=0}^n x_i y_i
\]
\[
a \sum_{i=0}^n x_i + b (n+1) = \sum_{i=0}^n y_i
\]
Если же выбранная функция нелинейна, необходимо провести предварительную линеаризацию, введя 
соответствующие замены, а затем, после вычисления коэффициентов, вернуться к ним.
Например, для функции $z_9$ нужно предварительно перейти к обратной величине: $\frac{1}{z_9} = aln(x) + b$. Элементы 
системы уравнений метода наименьших квадратов в таком случае будут состоять из сумм величин $ln(x_i)$ и $\frac{1}{y_i}$.




\end{justify}
\pagebreak

\section{Практическая реализация}
Листинг 1 — реализация программы
\begin{minted}{python}
from matplotlib import pyplot as plt
import numpy as np


def plot_fun(x_vals, y_vals, z, z_9):
	x_vals_z = np.linspace(1, 5, 1000)
	z_vals = z(x_vals_z)
	z_9_vals = z_9(x_vals_z)
	
	plt.plot(x_vals, y_vals, 'o', label='Точки таблично заданной функции')
	plt.plot(x_vals_z, z_vals, label='Функция, подобранная вручную')
	plt.plot(x_vals_z, z_9_vals, label='Функция, найденная программно')
	plt.axhline(0, color='gray', linestyle='--')  # ось X
	plt.axvline(0, color='gray', linestyle='--')  # ось Y
	plt.title("Точки из таблицы и z(x)")
	plt.xlabel("x")
	plt.ylabel("z(x)")
	plt.grid(True)
	plt.legend(loc='best')
	plt.show()


def calculate_vals(x_vals, y_vals, z):
	x_a = (x_vals[0] + x_vals[-1]) / 2
	y_a = (y_vals[0] + y_vals[-1]) / 2
	x_g = np.sqrt(x_vals[0] * x_vals[-1])
	y_g = np.sqrt(y_vals[0] * y_vals[-1])
	x_h = 2 / (1 / x_vals[0] + 1 / x_vals[-1])
	y_h = 2 / (1 / y_vals[0] + 1 / y_vals[-1])
	z_a = z(x_a)
	z_g = z(x_g)
	z_h = z(x_h)
	
	print(f'x_a = {x_a:.1f}, y_a = {y_a:.1f}, \n'
		f'x_g = {x_g:.1f}, y_g = {y_g:.1f}, \n'
		f'x_h = {x_h:.1f}, y_h = {y_h:.1f}, \n'
		f'z_a = {z_a:.1f}, z_g = {z_g:.1f}, z_h = {z_h:.1f} \n')
	
	delta_1 = abs(z_a - y_a)
	delta_2 = abs(z_g - y_g)
	delta_3 = abs(z_a - y_g)
	delta_4 = abs(z_g - y_a)
	delta_5 = abs(z_h - y_a)
	delta_6 = abs(z_a - y_h)
	delta_7 = abs(z_h - y_h)
	delta_8 = abs(z_h - y_g)
	delta_9 = abs(z_g - y_h)
	
	print(f'delta 1 ... 9: \n'
		f'{delta_1:.1f}, {delta_2:.1f}, {delta_3:.1f}, \n'
		f'{delta_4:.1f}, {delta_5:.1f}, {delta_6:.1f}, \n'
		f'{delta_7:.1f}, {delta_8:.1f}, {delta_9:.1f}')
	print(f'delta min: {min([delta_1, delta_2, delta_3,
		delta_4, delta_5, delta_6,
		delta_7, delta_8, delta_9]):.1f} \n')
	# delta_9 -- минимальное


def min_quad(x_vals, y_vals):
	u = np.log(x_vals)          # ln x_i
	w = 1 / np.array(y_vals)    # 1 / y_i
	
	n  = len(x_vals)
	S1 = np.sum(u * u)          # sum u_i^2
	S2 = np.sum(u)              # sum u_i
	S3 = np.sum(u * w)          # sum u_i * w_i
	S4 = np.sum(w)              # sum w_i
	
	det = S1 * n - S2 ** 2
	if abs(det) < 1e-12:
		raise ValueError("Система вырождается. Где-то ошибка")
	
	a = (S3 * n - S2 * S4) / det
	b = (S1 * S4 - S2 * S3) / det
	return a, b


def sko(x_vals, y_vals, z):
	n = len(x_vals)
	summ = 0
	for i in range(n):
		summ += (z(x_vals[i]) - y_vals[i]) ** 2
	return np.sqrt(summ) / np.sqrt(n - 1)


def main():
	x_vals = np.linspace(1, 5, 9)
	y_vals = [1.24, 1.74, 1.61, 2.16, 3.06, 2.88, 4.53, 5.40, 7.07]
	
	z = lambda x: 0.7 * np.exp(x * 0.44)
	
	calculate_vals(x_vals, y_vals, z) # delta_9 оказалось минимальным
	
	# z_9 = 1/(alnx + b)
	# 1 / z_9 = alnx + b
	
	a, b = min_quad(x_vals, y_vals)
	z_9 = lambda x: 1 / (a * np.log(x) + b)
	
	plot_fun(x_vals, y_vals, z, z_9)
	
	print(f'СКО: {sko(x_vals, y_vals, z_9):.3}')


if __name__ == '__main__':
	main()



\end{minted}

Перед выполнением работы была вручную подобрана функция $z(x) = 0.7 e^{0.44x}$.
Сначала в программе проводится вычисление значений $x_a$, $x_g$, $x_h$, $y_a$, $y_g$, $y_h$, $z(x_a)$, $z(x_g)$, $z(x_h)$, 
а также $\delta_1$, ..., $\delta_9$, $\delta_k = min(\delta_i)$ и вывод их на экран. Оказалось, что 
$\delta_9$ является минимальной среди всех $\delta_i$, поэтому в качестве аппроксимационной выбрана функция $z_9$.

После того, как был определён нужный вид функции, была реализована функция $min\_quad()$ вычисления её коэффициентов 
методом наименьших квадратов. В ней предварительно проводится линеаризация функции 
$z_9(x)$: $ln(x_i)$ заменяется на $u_i$, $\frac{1}{y_i}$ -- на $w_i$. Система уравнений решается методом Крамера, 
после чего по найденным коэффициентам определяется функция $z_9(x)$. На экран также выводится среднеквадратичное отклонение, 
вычисляемое в функции $sko()$ и формулы для подобранной вручную и найденной аппрорксимационных функций.

Результат, который вывод программа, приведён на рисунке 1. Графики исходной и найденной программно аппроксимационных функций 
представлены на рисунке 2. Жёлтым цветом показана функция, подобранная вручную, а 
зелёным -- функция, найденная программно.


\begin{figure}[H]
	
	\centering
	
	\includegraphics[width=0.8\linewidth]{1.png}
	\captionsetup{justification=centering}
	\caption{Результат выполнения программы}
	
	\label{fig:mpr}
	
\end{figure}

\begin{figure}[H]
	
	\centering
	
	\includegraphics[width=0.8\linewidth]{2.png}
	\captionsetup{justification=centering}
	\caption{Построенный график функций}
	
	\label{fig:mpr}
	
\end{figure}

\pagebreak

\section{Вывод}
В данной лабораторной работе была реализована аппроксимация методом наименьших квадратов для двупараметрических моделей.
Заданная таблично функция, исходная аппроксимационная функция, найденная вручную, а также найденная программно аппроксимационная функция выводятся на график. Также на экран выводится итоговое среднеквадратичное отклонение для 
найденной функции и формулы, описывающие исходную и найденную функции.
\end{document}
