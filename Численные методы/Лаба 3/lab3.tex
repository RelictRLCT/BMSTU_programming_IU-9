\documentclass[a4paper, 14pt]{extarticle}

 
% Поля
%--------------------------------------
\usepackage{minted}
\usepackage{xcolor}
\usepackage{geometry}
\usepackage{float}
\usepackage{ragged2e}
\geometry{a4paper,tmargin=2cm,bmargin=2cm,lmargin=3cm,rmargin=1cm}
%--------------------------------------

% Настройка minted
%--------------------------------------
\definecolor{bg}{rgb}{0.95,0.95,0.95} % Цвет фона
\definecolor{commentgreen}{rgb}{0,0.5,0} % Цвет комментариев

\setminted{
	frame=lines,             % рамка сверху и снизу
	framesep=2mm,            % отступ от рамки до кода
	baselinestretch=1.2,     % межстрочный интервал
	fontsize=\footnotesize,  % размер шрифта
	linenos,                 % нумерация строк
	breaklines,              % перенос строк, если они длинные
	%bgcolor=bg,              % цвет фона
	tabsize=4,               % размер табуляции
	numbersep=5pt,           % отступ номеров строк от кода
	escapeinside=||,         % возможность вставить LaTeX внутрь кода
}
%--------------------------------------

%Russian-specific packages
%--------------------------------------
\usepackage[T2A]{fontenc}
\usepackage[utf8]{inputenc} 
\usepackage[english, main=russian]{babel}
%--------------------------------------

\usepackage{textcomp}

% Красная строка
%--------------------------------------
\usepackage{indentfirst}               
%--------------------------------------             


%Graphics
%--------------------------------------
\usepackage{graphicx}
\graphicspath{ {./images/} }
\usepackage{wrapfig}
%--------------------------------------

% Полуторный интервал
%--------------------------------------
\linespread{1.3}                    
%--------------------------------------

%Выравнивание и переносы
%--------------------------------------
% Избавляемся от переполнений
\sloppy
% Запрещаем разрыв страницы после первой строки абзаца
\clubpenalty=10000
% Запрещаем разрыв страницы после последней строки абзаца
\widowpenalty=10000
%--------------------------------------

%Списки
\usepackage{enumitem}

%Подписи
\usepackage{caption} 

%Гиперссылки
\usepackage{hyperref}

\hypersetup {
	unicode=true
}

%Рисунки
%--------------------------------------
\DeclareCaptionLabelSeparator*{emdash}{~--- }
\captionsetup[figure]{labelsep=emdash,font=onehalfspacing,position=bottom}
%--------------------------------------

\usepackage{newtxtext,newtxmath}
\usepackage{amsmath}
\usepackage{color}
\usepackage{listings}
\lstset{
  belowcaptionskip=1\baselineskip,
  breaklines=true,
  frame=L,
  xleftmargin=\parindent,
  language=C++,
  showstringspaces=false,
  basicstyle=\footnotesize\ttfamily,
  keywordstyle=\bfseries\color{blue},
  commentstyle=\itshape\color{purple},
  identifierstyle=\color{black},
  stringstyle=\color{red},
  extendedchars=\true,
}

%--------------------------------------
%			НАЧАЛО ДОКУМЕНТА
%--------------------------------------

\begin{document}

%--------------------------------------
%			ТИТУЛЬНЫЙ ЛИСТ
%--------------------------------------
\begin{titlepage}
\thispagestyle{empty}
\newpage


%Шапка титульного листа
%--------------------------------------
\vspace*{-60pt}
\hspace{-65pt}
\begin{minipage}{0.3\textwidth}
\hspace*{-20pt}\centering
\includegraphics[width=\textwidth]{emblem.png}
\end{minipage}
\begin{minipage}{0.67\textwidth}\small \textbf{
\vspace*{-0.7ex}
\hspace*{-6pt}\centerline{Министерство науки и высшего образования Российской Федерации}
\vspace*{-0.7ex}
\centerline{Федеральное государственное автономное образовательное учреждение }
\vspace*{-0.7ex}
\centerline{высшего образования}
\vspace*{-0.7ex}
\centerline{<<Московский государственный технический университет}
\vspace*{-0.7ex}
\centerline{имени Н.Э. Баумана}
\vspace*{-0.7ex}
\centerline{(национальный исследовательский университет)>>}
\vspace*{-0.7ex}
\centerline{(МГТУ им. Н.Э. Баумана)}}
\end{minipage}
%--------------------------------------

%Полосы
%--------------------------------------
\vspace{-25pt}
\hspace{-35pt}\rule{\textwidth}{2.3pt}

\vspace*{-20.3pt}
\hspace{-35pt}\rule{\textwidth}{0.4pt}
%--------------------------------------

\vspace{1.5ex}
\hspace{-35pt} \noindent \small ФАКУЛЬТЕТ\hspace{80pt} <<Информатика и системы управления>>

\vspace*{-16pt}
\hspace{47pt}\rule{0.83\textwidth}{0.4pt}

\vspace{0.5ex}
\hspace{-35pt} \noindent \small КАФЕДРА\hspace{50pt} <<Теоретическая информатика и компьютерные технологии>>

\vspace*{-16pt}
\hspace{30pt}\rule{0.866\textwidth}{0.4pt}
  
\vspace{11em}

\begin{center}
\Large {\bf Лабораторная работа № 3} \\ 
\large {\bf по курсу <<Численные методы>>} \\ 
\end{center}\normalsize

\vspace{8em}


\begin{flushright}
  {Студент группы ИУ9-62Б Нащёкин Н.Д.\hspace*{15pt} \\
  \vspace{2ex}
  Преподаватель: Домрачева А.Б.\hspace*{15pt}}
\end{flushright}

\bigskip

\vfill
 

\begin{center}
\textsl{Москва, 2025}
\end{center}
\end{titlepage}
%--------------------------------------
%		КОНЕЦ ТИТУЛЬНОГО ЛИСТА
%--------------------------------------

\renewcommand{\ttdefault}{pcr}

\setlength{\tabcolsep}{3pt}
\newpage
\setcounter{page}{2}

\section{Задача}
\begin{justify}
1. Найти численно с погрешностью $\varepsilon = 0.001$ решение задачи Коши дифференциального уравнения второго 
порядка 
\[
y'' + py' + qy = f(x), y(0) = y_0, y'(0) = y'_0 
\]
на отрезке [0, 1], приведя его к СОДУ первого порядка. Использовать классический метод Рунге-Кутта.\\
2. Найти точное решение дифференциального уравнения.\\
3. Сравнить приближённое и точное решения на каждом шаге вычислений.


\end{justify}
\pagebreak

\section{Основная теория}
\begin{justify}

\noindent\large {Метод Рунге-Кутта}\\ \normalsize

Метод Рунге-Кутта применяют для решения задачи Коши для системы обыкновенных дифференциальных уравнений (СОДУ)
первого порядка:

\begin{equation*}
	\begin{aligned}
		y_1' &= f_1(x, y_1, y_2, \ldots, y_n), \\
		y_2' &= f_2(x, y_1, y_2, \ldots, y_n), \\
		&\vdots \\
		y_n' &= f_n(x, y_1, y_2, \ldots, y_n).
	\end{aligned}
\end{equation*}
на отрезке $[x_0, x_{\text{end}}]$ с начальными условиями 
\[
y_1(x_0) = y_{01}, \quad \ldots, \quad y_n(x_0) = y_{0n}.
\]
Требуется приближённо найти решения системы 
\[
y_1(x_{\text{end}}), \quad \ldots, \quad y_n(x_{\text{end}})
\]
в конечной точке отрезка.
Запишем систему в векторной форме:
\[
\mathbf{y}' = \mathbf{f}(x, \mathbf{y}), \quad \mathbf{y}(x_0) = \mathbf{y}_0.
\]
Здесь $\mathbf{y} = (y_1(x), \ldots, y_n(x))$ и $\mathbf{y}'$ — векторы неизвестных функций и их производных; 
$\mathbf{f} = (f_1(x, \mathbf{y}), \ldots, f_n(x, \mathbf{y}))$ — вектор правых частей; 
а $\mathbf{y}_0 = (y_{01}, \ldots, y_{0n})$ — вектор начальных условий.

Метод Рунге-Кутта позволяет последовательно, зная решение системы в некоторой точке $x$ отрезка $[x_0, x_{\text{end}}]$,
продвигаться на шаг $h$, то есть приближённо искать решение в точке $x + h$, затем в точке $x + 2h$ и так далее, пока 
не доберёмся до $x_{\text{end}}$.

Сам метод заключается в нахождении вектор-коэффициентов $k_1, k_2, k_3$ и $k_4$ по следующим формулам:

\begin{equation*}
	\begin{aligned}
		k_1 &= \mathbf{f}(x, \mathbf{y}), \\
		k_2 &= \mathbf{f}(x + \frac{h}{2}, \mathbf{y} + \frac{hk_1}{2}), \\
		k_3 &= \mathbf{f}(x + \frac{h}{2}, \mathbf{y} + \frac{hk_2}{2}), \\
		k_4 &= \mathbf{f}(x + h, \mathbf{y} + hk_3).
	\end{aligned}
\end{equation*}
и построении очередного приближения к решению СОДУ в точке $x + h$ по формуле:
\[
\mathbf{y}(x + h) \approx \mathbf{y}_h = \mathbf{y} + \frac{h}{6}(k_1 + 2(k_2 + k_3) + k_4).
\]
Этот метод имеет четвёртый порядок точности, на каждом частичном отрезке:
\[
||\mathbf{y}(x + h) - \mathbf{y}_h|| = \max_{1 \le i \le n} \left| y_i(x + h) - y_{h, i} \right| \leq Ch^5
\]
Поэтому при суммировании, так как число частичных отрезков равно $\frac{b - a}{h}$, получим $C(b - a)h^4$ в правой части 
неравенства. 

Кроме того, существует алгоритм автоматического управления длиной шага $h$, обеспечивающий погрешность на каждом шаге не более $\varepsilon$:

1. Если начальная длина шага была выбрана равной $h$, то проводится вычисление векторов решений для шагов длины $h$ и длины $\frac{h}{2}$.


2. Вычисляется погрешность по правилу Рунге практической оценки погрешности:
\[
\text{err} = \frac{\left\| \mathbf{y}_h - \mathbf{y}_{h/2} \right\|}{2^p - 1},
\]
где $p$ — порядок точности используемого метода (равен 4 в случае с методом Рунге-Кутта). Здесь для вектора $\mathbf{y}_{h/2}$
при вычислении нормы берётся каждая вторая координата.

3. Величина $\text{err}$ сравнивается с $\varepsilon$, что позволяет вычислить оптимальную длину шага:
\[
h_{\text{opt}} = h \left( \frac{\varepsilon}{\text{err}} \right)^{\frac{1}{p+1}}.
\]

4. Если $\text{err} \le \varepsilon$, то два вычисления $\mathbf{y}_h$ и $\mathbf{y}_{h/2}$ считаются 
принятыми и приближённым решением считается вектор $\mathbf{y}_{h/2}$. В противном случае оба шага отбрасываются 
и вычисления повторяются с длиной шага $h_{\text{new}} = 0.9 h_{\text{opt}}$.

\end{justify}
\pagebreak

\section{Практическая реализация}
Листинг 1 — реализация программы
\begin{minted}{python}
import math
from copy import deepcopy

import numpy as np

# y'' + p y' + q y = f(x)
# Вариант 22: p = -1, q = 0, f(x) = 3, y0 = 0, y0' = 2
# y'' - y' = 3
#
# Система: u' = v
#          v' = v + 3
# u(0) = 0, v(0) = 2
#
# Точное решение: u(1) = 5.59141 , v(1) = 10.59141
u_1_true = 5.59141
v_1_true = 10.59141

eps = 0.001


def u(t):
	return 5 * math.exp(t) - 3 * t - 5


def v(t):
	return 5 * math.exp(t) - 3


def f(x, y):
	u, v = y
	f1 = v       # u' = v
	f2 = v + 3   # v' = v + 3
	return np.array([f1, f2])


def runge_kutta_step(f, x, y, h):
	k1 = f(x, y)
	k2 = f(x + h / 2, y + h * k1 / 2)
	k3 = f(x + h / 2, y + h * k2 / 2)
	k4 = f(x + h, y + h * k3)
	return y + (h / 6) * (k1 + 2 * (k2 + k3) + k4)


def norm_one(a, b):
	max_val = 0
	for i in range(len(a)):
		cur_val = abs(a[i] - b[i])
		if cur_val > max_val:
			max_val = cur_val
	return max_val


def norm_full(a, b):
	max_val = 0
	for i in range(len(a)):
		try:
			cur_val = norm_one(a[i], b[2 * i])
		except IndexError:
			cur_val = 0
		if cur_val > max_val:
			max_val = cur_val
	return max_val


def runge_rule(y_h, y_2h, p):
	return norm_full(y_h, y_2h) / (2 ** p - 1)


def get_new_step(h, err, p):
	try:
		return 0.9 * h * (eps / err) ** (1 / (p + 1))
	except ZeroDivisionError:
		return 0.9 * h


def runge_solve(f, y0, x0, xend, h):
	x = x0
	y = y0
	y_vals = []
	x_vals = []
	y_real = []
	while x < xend:
		#print(h, x)
		if x + h - xend >= 0:
			y_vals.append(y)
			x_vals.append(x)
			y_real.append([u(x), v(x)])
			h = xend - x  # чтобы не перепрыгнуть через xend
			x = x + h
			y = runge_kutta_step(f, x, y, h)
			y_vals.append(y)
			x_vals.append(x)
			y_real.append([u(x), v(x)])
			break
		y_vals.append(y)
		x_vals.append(x)
		y_real.append([u(x), v(x)])
		x = x + h
		y = runge_kutta_step(f, x, y, h)
	return np.array(y_vals), np.array(y_real), np.array(x_vals)


def runge_kutta_with_autostep(f, y0, x0, xend, h):
	# Порядок метода
	p = 4
	
	y_h, _, _ = runge_solve(f, y0, x0, xend, h)
	y_h_2, y_r, xs = runge_solve(f, y0, x0, xend, h / 2)
	err = runge_rule(y_h, y_h_2, p)
	
	while err > eps:
		h = get_new_step(h , err, p)
		# print(h, err)
		y_h, _, _ = runge_solve(f, y0, x0, xend, h)
		y_h_2, y_r, xs = runge_solve(f, y0, x0, xend, h / 2)
		err = runge_rule(y_h, y_h_2, p)
	
	return np.array(y_h_2), np.array(y_r), np.array(xs)


def main():
	x0 = 0
	xend = 1
	
	y0 = np.array([0.0, 2.0]) # Н.у. u(0)=0, v(0)=2
	h = (xend - x0) / 2.0
	
	y, y_real, x = runge_kutta_with_autostep(f, y0, x0, xend, h)
	
	for i in range(len(y)):
		print(f'x = {x[i]:.8f}, [u(x), v(x)] приближённо = [{y[i][0]:.8f}, {y[i][1]:.8f}], '
			f'точное значение [u(x), v(x)] = [{y_real[i][0]:.8f}, {y_real[i][1]:.8f}], '
			f'абсолютная погрешность: {norm_one(y_real[i], y[i]):.8f}')
	print()
	
	print(f'eps = {eps}')
	print(f"Точные значения: u(1) = {u(xend):.8f}, v(1) = {v(xend):.8f}")
	print(f"Вычислено методом Рунге-Кутта: u(1) = {y[-1][0]:.8f}, v(1) = {y[-1][1]:.8f}")


if __name__ == '__main__':
	main()


\end{minted}

Мой вариант -- 22. По условию необходимо найти приближённое решение дифференциального уравнения $y'' - y' = 3$ с 
начальными условиями $y_0 = 0$, $y'_0 = 2$.
Для начала необходимо привести уравнение второго порядка к системе уравнений первого порядка. Вводим замены:
$u = y, v = y'$. Исходя из замен, получим систему:
\begin{equation*}
	\begin{aligned}
		u' &= v \\
		v' &= v + 3
	\end{aligned}
\end{equation*}
и начальные условия: $u(0) = 0, v(0) = 2$. Эта система приближённо решается с использованием описанного метода Рунге-Кутта.

Изначально был реализован метод Рунге-Кутта без автоматического управления шагом: на каждой итерации длина шага просто 
делилась пополам. Затем было добавлено управление шагом для этого метода. На рисунке 
1 представлен вывод результата работы программы. 

\begin{figure}[H]
	
	\centering
	
	\includegraphics[width=0.8\linewidth]{1.png}
	\captionsetup{justification=centering}
	\caption{Результат выполнения программы}
	
	\label{fig:mpr}
	
\end{figure}

\pagebreak

\section{Вывод}
В данной лабораторной работе был реализован метод Рунге-Кутта, позволяющий приближённо находить решение задачи Коши для 
системы обыкновенных дифференциальных уравнений. Кроме того, были выведены результаты работы реализованного метода и точного решения на каждом шаге. Также было проведено сравнение работы метода без управления шагом и с использованием алгоритма автоматического управления шагом. При сравнении погрешность была установлена как $\varepsilon = 0.00001$. Был получен следующий результат: в первом случае итоговое количество точек разбиения отрезка оказалось равно $17$, тогда как во втором случае -- $12$, то есть алгоритм 
управления шагом уменьшил количество вычислений, сохранив точность. 
\end{document}
