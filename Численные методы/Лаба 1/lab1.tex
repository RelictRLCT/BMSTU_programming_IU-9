\documentclass[a4paper, 14pt]{extarticle}

 
% Поля
%--------------------------------------
\usepackage{minted}
\usepackage{xcolor}
\usepackage{geometry}
\usepackage{float}
\geometry{a4paper,tmargin=2cm,bmargin=2cm,lmargin=3cm,rmargin=1cm}
%--------------------------------------


%Russian-specific packages
%--------------------------------------
\usepackage[T2A]{fontenc}
\usepackage[utf8]{inputenc} 
\usepackage[english, main=russian]{babel}
%--------------------------------------

\usepackage{textcomp}

% Красная строка
%--------------------------------------
\usepackage{indentfirst}               
%--------------------------------------             


%Graphics
%--------------------------------------
\usepackage{graphicx}
\graphicspath{ {./images/} }
\usepackage{wrapfig}
%--------------------------------------

% Полуторный интервал
%--------------------------------------
\linespread{1.3}                    
%--------------------------------------

%Выравнивание и переносы
%--------------------------------------
% Избавляемся от переполнений
\sloppy
% Запрещаем разрыв страницы после первой строки абзаца
\clubpenalty=10000
% Запрещаем разрыв страницы после последней строки абзаца
\widowpenalty=10000
%--------------------------------------

%Списки
\usepackage{enumitem}

%Подписи
\usepackage{caption} 

%Гиперссылки
\usepackage{hyperref}

\hypersetup {
	unicode=true
}

%Рисунки
%--------------------------------------
\DeclareCaptionLabelSeparator*{emdash}{~--- }
\captionsetup[figure]{labelsep=emdash,font=onehalfspacing,position=bottom}
%--------------------------------------

\usepackage{newtxtext,newtxmath}
\usepackage{amsmath}
\usepackage{color}
\usepackage{listings}
\lstset{
  belowcaptionskip=1\baselineskip,
  breaklines=true,
  frame=L,
  xleftmargin=\parindent,
  language=C++,
  showstringspaces=false,
  basicstyle=\footnotesize\ttfamily,
  keywordstyle=\bfseries\color{blue},
  commentstyle=\itshape\color{purple},
  identifierstyle=\color{black},
  stringstyle=\color{red},
  extendedchars=\true,
}

%--------------------------------------
%			НАЧАЛО ДОКУМЕНТА
%--------------------------------------

\begin{document}

%--------------------------------------
%			ТИТУЛЬНЫЙ ЛИСТ
%--------------------------------------
\begin{titlepage}
\thispagestyle{empty}
\newpage


%Шапка титульного листа
%--------------------------------------
\vspace*{-60pt}
\hspace{-65pt}
\begin{minipage}{0.3\textwidth}
\hspace*{-20pt}\centering
\includegraphics[width=\textwidth]{emblem.png}
\end{minipage}
\begin{minipage}{0.67\textwidth}\small \textbf{
\vspace*{-0.7ex}
\hspace*{-6pt}\centerline{Министерство науки и высшего образования Российской Федерации}
\vspace*{-0.7ex}
\centerline{Федеральное государственное автономное образовательное учреждение }
\vspace*{-0.7ex}
\centerline{высшего образования}
\vspace*{-0.7ex}
\centerline{<<Московский государственный технический университет}
\vspace*{-0.7ex}
\centerline{имени Н.Э. Баумана}
\vspace*{-0.7ex}
\centerline{(национальный исследовательский университет)>>}
\vspace*{-0.7ex}
\centerline{(МГТУ им. Н.Э. Баумана)}}
\end{minipage}
%--------------------------------------

%Полосы
%--------------------------------------
\vspace{-25pt}
\hspace{-35pt}\rule{\textwidth}{2.3pt}

\vspace*{-20.3pt}
\hspace{-35pt}\rule{\textwidth}{0.4pt}
%--------------------------------------

\vspace{1.5ex}
\hspace{-35pt} \noindent \small ФАКУЛЬТЕТ\hspace{80pt} <<Информатика и системы управления>>

\vspace*{-16pt}
\hspace{47pt}\rule{0.83\textwidth}{0.4pt}

\vspace{0.5ex}
\hspace{-35pt} \noindent \small КАФЕДРА\hspace{50pt} <<Теоретическая информатика и компьютерные технологии>>

\vspace*{-16pt}
\hspace{30pt}\rule{0.866\textwidth}{0.4pt}
  
\vspace{11em}

\begin{center}
\Large {\bf Лабораторная работа № 1} \\ 
\large {\bf по курсу <<Численные методы>>} \\ 
\end{center}\normalsize

\vspace{8em}


\begin{flushright}
  {Студент группы ИУ9-62Б Нащекин Н.Д.\hspace*{15pt} \\
  \vspace{2ex}
  Преподаватель: Домрачева А.Б.\hspace*{15pt}}
\end{flushright}

\bigskip

\vfill
 

\begin{center}
\textsl{Москва, 2025}
\end{center}
\end{titlepage}
%--------------------------------------
%		КОНЕЦ ТИТУЛЬНОГО ЛИСТА
%--------------------------------------

\renewcommand{\ttdefault}{pcr}

\setlength{\tabcolsep}{3pt}
\newpage
\setcounter{page}{2}

\section{Задача}
\begin{flushleft}
• Протабулировать функцию $f(x)$ на отрезке $[a, b]$ с шагом $h = \frac{b - a}{32}$
и распечатать таблицу $(x_i, y_i), i = 0, ... n$. Для полученных узлов 
$(x_i, y_i), i = 0, ... n$ построить кубический сплайн (распечатать массивы $a, b, c$ и $d$).
Вычислить значения $f(x)$ в точках $x_i = a + (i-\frac{1}{2})h$, $i = 0, ... n$.
Вычислить значения оригинальной функции и сплайна в произвольной точке.


\end{flushleft}
\pagebreak

\section{Основная теория}
\begin{flushleft}
	
Если задана функция $f(x)$, значения которой известны в точках $(x_i, y_i), i = 0, ... n$,
то интерполяционной называется функция $y = \varphi(x)$, проходящая через эти точки (они 
называются узлами интерполирования). В промежуточной точке равенство $f(x) \approx \varphi(x)$
выполняется лишь с некоторой погрешностью.

Одним из способов построения интерполяционной функции является нахождение сплайнов.
Сплайн $k$-го порядка с дефектом def — функция, проходящая через все узлы 
$(x_i, y_i), i = 0, ... n$, которая является многочленом $k$-й степени на каждом отрезке разбиения
$[x_i, x_{i+1}]$ (такая функция также называется кусочно-полиномиальной) и имеет $(k - def)$ 
непрерывных производных на отрезке $[x_0, x_n]$.

Сплайн третьего порядка с дефектом 1 можно отыскать в виде:
\[
\begin{aligned}
	S_i(x) &= a_i + b_i(x - x_i) + c_i(x - x_i)^2 + d_i(x - x_i)^3, \\
	x &\in [x_i, x_{i+1}], \quad i = 0, \dots, n-1
\end{aligned}
\]

Условия на частные многочлены:

\[
S_i(x_i) = y_i, \quad i = 0, \dots, n-1; \quad S_{n-1}(x_n) = y_n
\]

(сплайн проходит через все узлы интерполирования). Также 

\[
S_{i-1}(x_i) = S_i(x_i); \quad S'_{i-1}(x_i) = S'_i(x_i);
\]

\[
S''_{i-1}(x_i) = S''_i(x_i), \quad i = 0, \dots, n-1
\]

(непрерывность сплайна и его первых двух производных в промежуточных узлах) и 

\[
S''_0(x_0) = 0; \quad S''_{n-1}(x_n) = 0
\]

(условия гладкости на краях).  

Эти условия приводят к трехдиагональной СЛАУ относительно коэффициентов \( c_i \):

\[
c_{i-1} + 4c_i + c_{i+1} = \frac{y_{i+1} - 2y_i + y_{i-1}}{h^2}, \quad i = 1, \dots, n-1
\]

\[
c_0 = 0, \quad c_{n} = 0
\]

где  
\[
h = x_{i+1} - x_i, \quad i = 0, \dots, n-1
\]
— постоянный шаг интерполирования.

Необходимо решить полученную систему (например, методом прогонки или методом Гаусса).

Остальные коэффициенты можно выразить через \( c_i \) по следующим формулам:

\[
\alpha_i = y_i, \quad i = 0, \dots, n-1;
\]

\[
b_i = \frac{y_{i+1} - y_i}{h} - \frac{h}{3} (c_{i+1} + 2c_i), \quad i = 0, \dots, n-2;
\]

\[
b_{n-1} = \frac{y_n - y_{n-1}}{h} - \frac{2}{3} h c_{n-1};
\]

\[
d_i = \frac{c_{i+1} - c_i}{3h}, \quad i = 0, \dots, n-2;
\]

\[
d_{n-1} = -\frac{c_{n-1}}{3h}.
\]
\end{flushleft}
\pagebreak

\section{Практическая реализация}
Листинг 1 — реализация программы
\begin{minted}[frame=lines,framesep=2mm,baselinestretch=1.2,fontsize=\footnotesize,linenos]{py}
import numpy as np
import pandas as pd
import matplotlib.pyplot as plt


def gauss_solve(A: list[list[float]], b: list[float]) -> list[float]:
	n = len(A)

	# k-й шаг
	for k in range(n - 1):
		# Строка с максимальным по модулю ведущим элементом в k-м столбце
		maxline = k
		for i in range(k + 1, n):
			if abs(A[i][k]) > abs(A[maxline][k]):
				maxline = i
	
		# Перестановка ak и amaxline
		if maxline != k:
			A[k], A[maxline] = A[maxline], A[k]
		# Перестановка bk и bmaxline
		b[k], b[maxline] = b[maxline], b[k]
		
		# Обнуление элементов ниже главного
		for i in range(k + 1, n):
			if A[k][k] == 0:
				raise ValueError("Нулевой ведущий элемент")
			m = A[i][k] / A[k][k]
			A[i][k] = 0.0
			for j in range(k + 1, n):
				A[i][j] -= m * A[k][j]
				b[i] -= m * b[k]

	# Обратный ход
	x = [0] * n
	for i in range(n - 1, -1, -1):
		s = sum(A[i][j] * x[j] for j in range(i + 1, n))
		if A[i][i] == 0:
			raise ValueError("На главной диагонали 0")
		x[i] = (b[i] - s) / A[i][i]
	
	return x


def spline_coeffs_find(
	x: list[float], y: list[float]
) -> tuple[list[int], list[int], list[float], list[int]]:
	n = len(x) - 1
	# Шаг (b - a) / 32:
	h: float = (x[n] - x[0]) / 32
	
	# A и b для трёхдиаг. системы относительно ci
	A = [[0] * (n + 1) for _ in range(n + 1)]
	rhs = [0] * (n + 1)
	
	# c0 = 0; c_n = 0
	A[0][0] = 1
	A[n][n] = 1
	rhs[0] = 0
	rhs[n] = 0
	
	# Строки трёхдиаг. системы
	# c{i-1} + 4ci + c{i+1} = (y{i+1} - 2yi + y{i-1}) / h^2
	for i in range(1, n):
		A[i][i - 1] = 1
		A[i][i] = 4
		A[i][i + 1] = 1
		
		rhs[i] = 3 * (y[i + 1] - 2 * y[i] + y[i - 1]) / (h**2)
	
	c_result = gauss_solve(A, rhs)
	
	a_vals = [0] * n
	b_vals = [0] * n
	d_vals = [0] * n
	
	for i in range(n):
		# ai = yi:
		a_vals[i] = y[i]
		
		# b_i = (y{i+1} - yi)/h - h/3 * (c{i+1} + 2ci)
		# d_i = (c{i+1} - ci) / (3 * h)
		if i != n - 1:
			b_vals[i] = (y[i + 1] - y[i]) / h - (
				c_result[i + 1] + 2 * c_result[i]
			) * h / 3
			d_vals[i] = (c_result[i + 1] - c_result[i]) / (3 * h)
		else:
			b_vals[i] = ((y[n] - y[n - 1]) / h) - 2 * h * c_result[n - 1] / 3
			d_vals[i] = -1 * (c_result[n - 1]) / (3 * h)
	
	return a_vals, b_vals, c_result, d_vals


def cubic_spline_eval(
	x: list[float],
	a: list[float],
	b: list[float],
	c: list[float],
	d: list[float],
	X: float,
):
	# S_i(X) = a[i] + b[i]*(X - x[i])
	#               + c[i]*(X - x[i])^2
	#               + d[i]*(X - x[i])^3
	n = len(x) - 1
	
	if X < x[0] or X > x[n]:
		return None
	
	i = 0
	while i < n - 1 and not (x[i] <= X < x[i + 1]):
		i += 1
	
	dx = X - x[i]
	return a[i] + b[i] * dx + c[i] * (dx**2) + d[i] * (dx**3)


def plot(xs: list[float], ys: list[float]) -> None:
	plt.figure(figsize=(8, 5))
	plt.plot(
		xs,
		ys,
		marker="o",
		linestyle="-",
		color="b",
	)
	
	plt.xlabel("X")
	plt.ylabel("Y")
	plt.grid(True)
	
	plt.show()


def make_full_graph(
	a: list[float],
	b: list[float],
	c: list[float],
	d: list[float],
	left: float,
	right: float,
	x_nodes: list[float],
	step: float,
) -> None:
	i = left
	xs = []
	ys = []
	while i <= right:
		xs.append(i)
		ys.append(cubic_spline_eval(x_nodes, a, b, c, d, i))
		i += step
	plot(xs, ys)


def calc_vals(
	a: list[float],
	b: list[float],
	c: list[float],
	d: list[float],
	x_nodes: list[float],
) -> None:
	for i in range(1, len(x_nodes) + 1):
		point = (i - 1 / 2) * ((x_nodes[len(x_nodes) - 1] - x_nodes[0]) / 32)
		f_val = f(point)
		phi_val = cubic_spline_eval(x_nodes, a, b, c, d, point)
		print(
			f"f({point}) = {f_val}, "
			f"phi({point}) = {phi_val}, "
			f"|f({point}) - phi({point})| = {abs(phi_val - f_val)}"
		)


def f(x: float):
	return 8 * np.sin(0.5 * x) + 4 * np.cos(0.3 * x) - 0.2 * x


def main():
	x_nodes = list(range(32))
	y_nodes = [f(x) for x in x_nodes]
	
	# Таблица значений
	df = pd.DataFrame({"x": x_nodes, "y": y_nodes})
	print(df.to_string(index=False))
	
	a, b, c, d = spline_coeffs_find(x_nodes, y_nodes)
	
	print(
		"Массивы коэффициентов:\na: ",
		list(map(float, a)),
		"\nb: ",
		list(map(float, b)),
		"\nc: ",
		list(map(float, c)),
		"\nd: ",
		list(map(float, d)),
	)
	
	# График известных значений
	plot(x_nodes, y_nodes)
	
	# График сплайна
	make_full_graph(a, b, c, d, 0, 32, x_nodes, 0.1)
	
	print("Значения в точках x_i = a + (i + 1/2)h:")
	calc_vals(a, b, c, d, x_nodes)
	
	X = float(input("Введите точку (от 0 до 31): "))
	S_X = cubic_spline_eval(x_nodes, a, b, c, d, X)
	print(f"f({X}) = {f(X)}, сплайн в точке {X} равен {S_X}")


if __name__ == "__main__":
	main()



\end{minted}
Для тестирования программы была выбрана функция $f(x) = 8sin(\frac{x}{2}) + 4sin(0.3x) - 0.2x$. На рисунке 1 
представлена часть вывода значений исходной и интерполяционной функций в промежуточных точках отрезка, 
а также абсолютной погрешности. Для наглядности работы, помимо пунктов, указанных в задании, с использованием библиотеки matplotlib были также выведены графики исходной функции в точках 0...31 и интерполяционной функции в тех же точках, а также 
в промежуточных значениях. На рисунке 2 представлен график исходной функции (для наглядности точки соединяются отрезками). 
На рисунке 3 представлен график интерполяционной функции.

\begin{figure}[H]
	
	\centering
	
	\includegraphics[width=0.8\linewidth]{3.png}
	\captionsetup{justification=centering}
	\caption{Часть вывода значений $f(x)$, $\varphi(x)$ и $|f(x) - \varphi(x)|$ в промежуточных точках 
		отрезка $[0, 31]$}
	
	\label{fig:mpr}
	
\end{figure}

\begin{figure}[H]

\centering

\includegraphics[width=0.8\linewidth]{1.png}
\captionsetup{justification=centering}
\caption{График исходной функции $f(x)$ в точках $x = 0, ... 31$}

\label{fig:mpr}

\end{figure}

\begin{figure}[H]

\centering

\includegraphics[width=0.8\linewidth]{2.png}
\captionsetup{justification=centering}
\caption{График интерполяционной функции $\varphi(x)$ в точках $x$ из промежутка $[0, 31]$ с шагом 0.1}

\label{fig:mpr}

\end{figure}

\pagebreak

\section{Вывод}
В данной лабораторной работе был реализован поиск интерполяционной функции (кубического 
сплайна). Система уравнений с неизвестными $c_i$ решается с помощью метода Гаусса, реализованного
в нулевой лабораторной работе. Затем по представленным формулам вычисляются коэффициенты многочленов, 
после чего они выводятся на экран. После этого выводится таблица значений функции в известных точках, а также значения сплайнов в промежуточных точках. Кроме того, вычисляется абсолютная погрешность значений интерполяционной функции в промежуточных точках. Для наглядности значения функции и сплайнов были также построены на графиках. Также в программе 
предлагается ввести произвольную точку для вычисления в ней значений исходной и интерполяционной функций. В ходе выполнения работы была исправлена ошибка в формулах методички для $d_{n-1}$. Поскольку $c_n = 0$, при подстановке этого значения в общую формулу получается формула для границы, использующая $c_{n-1}$, а не $c_n$. Из полученных графиков наглядно видно, что
сплайн построен корректно: интерполяционная функция точно равна начальной функции в узлах интерполирования, а в промежуточных
узлах приближает её с некоторой погрешностью.
\end{document}
